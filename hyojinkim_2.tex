\documentclass[11pt,a4paper,reqno]{amsart} 
\usepackage{amsmath,amscd,amssymb,latexsym} 
\usepackage{longtable} 

\numberwithin{equation}{section} 

\def\baselinestretch{1.2} 

\newtheorem*{main}{Main Theorem} 
\newtheorem{thm}{Theorem}[section] 
\newtheorem{lem}[thm]{Lemma} 
\newtheorem{prop}[thm]{Proposition} 
\newtheorem{defn}[thm]{Definition} 
\newtheorem{cor}[thm]{Corollary} 
\newtheorem{ex}[thm]{Example} 
\newtheorem{rmk}[thm]{Remark} 

\pagestyle{plain} 

\newenvironment{pf}{{\noindent \bf Proof:\ }}{\hfill $\Box$ \bigskip} 

\renewcommand{\baselinestretch}{1.10}

\textwidth=17cm \textheight=25cm

\addtolength{\topmargin}{-40pt} \addtolength{\oddsidemargin}{-2cm}
\addtolength{\evensidemargin}{-2cm}

\setlength{\unitlength}{1mm} 



\begin{document} 


\title{Homework 1 : Dedekind Domains; Factorization} 
\author{Hyojin Kim} 
\date{march 22, 2019} 
\maketitle 

\section{Definitions} 

In this section, we will introduce some basic definitions to know the notions of Dedekind domain and the ideal class group. 
Being a generation of the ring $\mathbb{Z} \subset \mathbb{Q}$, the ring of integers $\mathcal{O}_L$ in an algebraic number field $L$, is at the center of all our considerations.

\begin{defn}
A Noetherian, integrally closed integral domain, not equal to a field, in which every nonzero prime ideal is maximal is called a Dedekind domain.
\end{defn}

The Dedekind domains my be viewed as generalized principal ideal domains. 
Let $A$ be a principal ideal domain with field of fractions $K$, and $L \mid K$ is a finite field extension, 
then the integral closure $B$ of $A$ in $L$ is not a principal ideal domain in general,
but always a Dedekind domain.

\begin{defn}
For a Dedekind domain $A$, a fractional ideal of $A$ is a nonzero $A$-submodule $\mathfrak{a}$ of $K$ such that
\[
d\mathfrak{a}:= \{da|a\in \mathfrak{a} \}
\]
is contained in $A$ for some nonzero $d \in A$(or $K$),
i.e., it is a nonzero $A$-submodule of $K$ whose elements have a common denominator.
Note that a fractional ideal is not an ideal unless it is contained in $A$, 
we refer to the ideals in $A$ as integral ideals.
Every nonzero element $b$ of $K$ defines a fractional ideal $(b):=bA:=\{ba|a \in A \}$. 
A fractional ideal of this type is said to be principal.
\end{defn}


\begin{defn}
The quotient $Cl(A)=Id(A)/P(A)$ of $Id(A)$ by the subfroup of principal ideals is the ideal class group of $A$.
The class number of $A$ is the order of $Cl(A)$(when finite). 
In the case that $A$ is the ring of integers $\mathcal{O}_K$ in $K$ in a number field $K$, 
we often refer to $Cl({\mathcal{O}_K})$ as the ideal class group of $K$, and its order as the class number of $K$.
\end{defn}

The class number of $\mathbb{Q}[\sqrt{-m}]$ for $m$ positive and square-free is $1$ iff $m=1,2,3,7,11,19,43,67,163$.
$\mathbb{Z}[\sqrt{-5}]$ is not a principal ideal domain, and so can't have class number 1. 
In fact, it has class number 2.
Gauss showed that the class group of a quadratic field $\mathbb{Q}[\sqrt{d}]$ can have arbitrarily many cyclic factors of even order.

We defined an integral basis and the discriminant already. 
Any basis of the free abelian group $A$ (ring of algebraic integers) is called an integral basis of $K$. 
An integral basis is a basis of the vector space $K$ over $\mathbb{Q}$, since it has $n[K:\mathbb{Q}]$ elements.
The discriminant in $K|\mathbb{Q}$ of any integral basis is called the discriminant of the field $K$.

Let $d_K$ be the discriminant  of Quadratic field $K=\mathbb{Q}(\sqrt{d})$ where $d$ is a squrae-free integer.
Then $d_K = 4d$ if $d \equiv 2~or~3 \pmod 4$, and $d_K = d$ if $d \equiv 1  \pmod 4$.
\section{Properties}

 



\begin{thebibliography}{alpha} 
\bibitem{M} James. S. Milne, \emph{Algebraic Number Theory (v3.07)}, 2017. Available at www.jmilne.org/math/. 
\bibitem{S} P. Samuel, \emph{Algebraic Theory of Numbers}, traslated from the French by Allan J.Silberger, HERMANN, Paris, 1970.
\end{thebibliography} 

\end{document} 