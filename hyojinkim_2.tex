\documentclass[11pt,a4paper,reqno]{amsart} 
\usepackage{amsmath,amscd,amssymb,latexsym} 
\usepackage{longtable} 

\numberwithin{equation}{section} 

\def\baselinestretch{1.2} 

\newtheorem*{main}{Main Theorem} 
\newtheorem{thm}{Theorem}[section] 
\newtheorem{lem}[thm]{Lemma} 
\newtheorem{prop}[thm]{Proposition} 
\newtheorem{defn}[thm]{Definition} 
\newtheorem{cor}[thm]{Corollary} 
\newtheorem{ex}[thm]{Example} 
\newtheorem{rmk}[thm]{Remark} 

\pagestyle{plain} 

\newenvironment{pf}{{\noindent \bf Proof:\ }}{\hfill $\Box$ \bigskip} 

\renewcommand{\baselinestretch}{1.10}

\textwidth=17cm \textheight=25cm

\addtolength{\topmargin}{-40pt} \addtolength{\oddsidemargin}{-2cm}
\addtolength{\evensidemargin}{-2cm}

\setlength{\unitlength}{1mm} 



\begin{document} 


\title{Homework 2 : Dedekind Domains; Factorization} 
\author{Hyojin Kim} 
\date{april 5, 2019} 
\maketitle 
%I uploaded the wrong file but I don't know it is possible to edit so upload the right file now. 
%introduction
In this note and the later note, we define the notion of a Dedekind domain and prove that
ideals in Dedekind domains factor uniguely into products of prime ideals, and 
rings of integers in number fields are Dedekind domains.

\section{Definitions} 
We will introduce some basic definitions to know the notions of Dedekind domain and the ideal class group. 
Being a generation of the ring $\mathbb{Z} \subset \mathbb{Q}$, 
the ring of integers $\mathcal{O}_L$ in an algebraic number field $L$, is at the center of all our considerations.

\begin{defn}
A \textbf{discrete valuation ring} is a principal ideal domain with exactly one non-zero prime ideal.
\end{defn}

\begin{defn}
A Noetherian, integrally closed integral domain, not equal to a field, 
in which every nonzero prime ideal is maximal is called a \textbf{Dedekind domain}.
\end{defn}

The Dedekind domains may be viewed as generalized principal ideal domains. 
Let $A$ be a principal ideal domain with field of fractions $K$, and $L/K$ is a finite field extension, 
then the integral closure $B$ of $A$ in $L$ is not a principal ideal domain in general,
but always a Dedekind domain.

\begin{defn}
For a Dedekind domain $A$, a \textbf{fractional ideal} of $A$ is a nonzero $A$-submodule $\mathfrak{a}$ of $K$ such that
\[
d\mathfrak{a}:= \{da~|~a\in \mathfrak{a} \}
\]
is contained in $A$ for some nonzero $d \in A$(or $K$),
i.e., it is a nonzero $A$-submodule of $K$ whose elements have a common denominator.
Note that a fractional ideal is not an ideal unless it is contained in $A$, 
we refer to the ideals in $A$ as \textbf{integral} ideals.
Every nonzero element $b$ of $K$ defines a fractional ideal $(b):=bA:=\{ba|a \in A \}$. 
A fractional ideal of this type is said to be principal.
\end{defn}

\begin{defn}
The quotient $Cl(A)=Id(A)/P(A)$ of $Id(A)$ by the subgroup of principal ideals is the \textbf{ideal class group} of $A$.
The \textbf{class number} of $A$ is the order of $Cl(A)$(when finite). 
In the case that $A$ is the ring of integers $\mathcal{O}_K$ in $K$ in a number field $K$, 
we often refer to $Cl({\mathcal{O}_K})$ as the \textbf{ideal class group} of $K$, 
and its order as the \textbf{class number} of $K$.
\end{defn}

The class number of $\mathbb{Q}[\sqrt{-m}]$ for $m$ positive and square-free is $1$ iff $m=1,2,3,7,11,19,43,67,163$.
$\mathbb{Z}[\sqrt{-5}]$ is not a principal ideal domain, and so can't have class number 1. 
In fact, it has class number 2.
Gauss showed that the class group of a quadratic field $\mathbb{Q}[\sqrt{d}]$ can have arbitrarily many cyclic factors of even order.

We defined an integral basis and the discriminant already. 
Any basis of the free abelian group $A$ (ring of algebraic integers) is called an integral basis of $K$. 
An integral basis is a basis of the vector space $K$ over $\mathbb{Q}$, since it has $n[K:\mathbb{Q}]$ elements.
The discriminant in $K|\mathbb{Q}$ of any integral basis is called the discriminant of the field $K$.

Let $d_K$ be the discriminant  of Quadratic field $K=\mathbb{Q}(\sqrt{d})$ where $d$ is a squrae-free integer.
Then $d_K = 4d$ if $d \equiv 2~or~3 \pmod 4$, and $d_K = d$ if $d \equiv 1  \pmod 4$.

Recall that for an integral domain $A$ with field of fraction $K$, 
we can define a multiplicative subset ${S_\mathfrak{p}}=A\setminus{\mathfrak{p}}$ of $A$, and 
we write ${A_\mathfrak{p}}={S^{-1}_{\mathfrak{p}}}A$ where $\mathfrak{p}$ is a prime ideal. For example, 
\[
{\mathbb{Z}_{(p)}}=\{m/n\in{\mathbb{Q}}~|~p\nmid n \}
\]
and $\mathbb{Z}_(p)$ is a discrete valuation ring with $(p)$ as its unique nonzero prime ideal.
Generally, if $\mathfrak{p}$ is a prime ideal in $A$, then $A_{\mathfrak{p}}$ is a local ring 
because $\mathfrak{p}$ contains every prime ideal disjoint from $S_{\mathfrak{p}}$.
Note that the ring $A_\mathfrak{p}$ is a discrete valuation ring.

\section{Unique factorization of ideals}

We now prove that a proper nonzero ideal $\mathfrak{a}$ of a Dedekind domain $A$ can be factored uniguely into a product of prime ideals.
To prove the existence of the prime ideal factorization, we will use followings without proof.

\begin{lem}\label{3.8}
Let $A$ be a Noetherian ring; then every ideal of $\mathfrak{a}$ in $A$ contains a product of nonzero prime ideals.
\end{lem}

\begin{thm}\label{1.14}
Let ${\mathfrak{a}_1},\dots,{\mathfrak{a}_n}$ be ideals in a ring $A$, relatively prime in pairs. 
Then for any elements ${x_1},\dots,{x_n}$ of $A$, the congruences 
\[
x\equiv{x_i} \pmod{\mathfrak{a}_i}
\]
have a simultaneous solution $x \in A$; moreover, if $x$ is one solution, 
then the other solutions are the elements of the form $x+a$ with $a \in \cap{\mathfrak{a}_i}$, and $\cap {\mathfrak{a}_i} = \prod {\mathfrak{a}_i}$.
In other words, the natural maps give an exact sequence
\[
0 \to \mathfrak{a} \to A \to {\prod_{i=1}^{n} {A/{\mathfrak{a}_i}}} \to 0
\]
with $\mathfrak{a}=\cap{\mathfrak{a}_i}=\prod{\mathfrak{a}_i}$.
\end{thm}

\begin{lem}\label{3.9}
Let $A$ be a ring and let $\mathfrak{a}$ and $\mathfrak{b}$ be relatively prime ideals in $A$; for any $m$, $n \in \mathbb{N}$, $\mathfrak{a}^m$ and $\mathfrak{b}^n$ are relatively prime.
\end{lem}

\begin{lem}\label{3.10}
Let $\mathfrak{p}$ be a maximal ideal of a ring $A$, and let $\mathfrak{q}$ be the ideal it generates in $A_{\mathfrak{p}}$, $\mathfrak{q}=\mathfrak{p}A_{\mathfrak{p}}$. The map
\[
{a+\mathfrak{p}^m}\mapsto{a+\mathfrak{q}^m}:{A/{\mathfrak{p}^m}}\mapsto{{A_{\mathfrak{p}}}/{\mathfrak{q}^m}}
\]
is an isomorphism.
\end{lem}

According to above, the ideal $\mathfrak{a}$ of $A$ contains a product of nonzero prime ideals, 
\[
\mathfrak{b}={\mathfrak{p}_1}^{r_1} \cdots {\mathfrak{p}_m}^{r_m},
\]
where the $\mathfrak{p}_i$ are distinct, and there exist isomorphisms
\[
A/{\mathfrak{b}}=A/{{\mathfrak{p}_1}^{r_1} \cdots {\mathfrak{p}_m}^{r_m}}
 \simeq A/{{\mathfrak{p}^{r_1}_1}} \times \cdots \times A/{{\mathfrak{p}^{r_m}_m}}
 \simeq {A_{\mathfrak{p}_1}}/{{\mathfrak{q}^{r_1}_1}} \times \cdots \times {A_{\mathfrak{p}_m}}/{{\mathfrak{q}^{r_m}_m}}
\]
where $\mathfrak{q}_i = \mathfrak{p}_i A_{\mathfrak{p}_i}$ is the maximal ideal of $A_{\mathfrak{p}_i}$. 
Recall that the rings $A_{\mathfrak{p}_i}$ are all discrete valuation rings. 
$\mathfrak{a}/\mathfrak{b}$ corresponds to ${\mathfrak{q}^{s_1}_1}/{\mathfrak{q}^{r_1}_1} \times \cdots \times {\mathfrak{q}^{s_m}_m}/{\mathfrak{q}^{r_m}_m}$ for some $s_i \leq r_i$.
Since this ideal is also the isomorphic image of ${\mathfrak{p}_1}^{s_1} \cdots {\mathfrak{p}_m}^{s_m}$, $\mathfrak{a}={\mathfrak{p}_1}^{s_1} \cdots {\mathfrak{p}_m}^{s_m}$ in $A/{\mathfrak{b}}$.
Hence $\mathfrak{a}={\mathfrak{p}_1}^{s_1} \cdots {\mathfrak{p}_m}^{s_m}$ in $A$ since both contain $\mathfrak{b}$ and there is a one-to-one correspondence between the ideals of $A/\mathfrak{b}$ and the ideals of $A$ containing $\mathfrak{b}$.

Let $\mathfrak{a}={\mathfrak{p}_1}^{s_1} \cdots {\mathfrak{p}_m}^{s_m}={\mathfrak{p}_1}^{t_1} \cdots {\mathfrak{p}_m}^{t_m}$ be two factorizations after adding factors with zero exponent.
We have $\mathfrak{a}A_{\mathfrak{p}_i} = \mathfrak{q}^{s_i}_i = \mathfrak{q}^{t_i}_i$ where $\mathfrak{q}_i$ the maximal ideal in $A_{\mathfrak{p}_i}$. Therefore $s_i = t_i$ for all $i$.

Now we get the following theorem.

\begin{thm}\label{3.7}
Let $A$ be a Dedekind domain. Every proper nonzero ideal $\mathfrak{a}$ of $A$ can be written in the form
\[
\mathfrak{a}={{\mathfrak{p}_1}^{r_1}\cdots{\mathfrak{p}_n}^{r_n}}
\]
with the $\mathfrak{p}_i$ distinct prime ideals and the ${r_i}>0$; the $\mathfrak{p}_i$ and the ${r_i}$ are uniquely determined.
\end{thm}


\begin{thebibliography}{alpha} 
\bibitem{M} James. S. Milne, \emph{Algebraic Number Theory (v3.07)}, 2017. Available at www.jmilne.org/math/. 
\bibitem{S} P. Samuel, \emph{Algebraic Theory of Numbers}, traslated from the French by Allan J.Silberger, HERMANN, Paris, 1970.
\end{thebibliography} 

\end{document} 