\documentclass[a4paper,11pt]{article}


\usepackage{amsmath}
\usepackage{amssymb}
\usepackage{tikz-cd}
\usepackage{amsthm}
\usepackage{mathtools}
\usepackage{titling}
\usepackage[left=2cm,right=2cm,top=1cm,bottom=2cm,a4paper]{geometry}
\usepackage{mathrsfs}
\usepackage{calrsfs}



\begin{document}

\theoremstyle{plain}
\newtheorem{thm}{Theorem}[section]
\newtheorem{prp}[thm]{Proposition}
\newtheorem{lem}[thm]{Lemma}

\theoremstyle{definition}
\newtheorem{defn}[thm]{Definition}
\newtheorem{exm}[thm]{Example}
\newtheorem{nota}[thm]{Notation}

\theoremstyle{remark}
\newtheorem{rem}[thm]{Remark}



\font\myfont=cmr12 at 20pt

\title{\vspace{-5ex} \myfont The Tate Curve}
\author{Jemin You}
\date{\vspace{-5ex}}
\maketitle

\setcounter{section}{-1}

We discuss the Tate curve, and how it is utilized to compute the cusps.
The 0th section is taken from the article of Katz in `Modular Functions of One Variable III', Springer LNM 350, while the others are from the book of Katz and Mazur.

This week's notes is going to be a preparation for the next week's one, where we will calculate the cusps for one of our moduli problems.

\section{The Tate Curve}

Consider a holomorphic function $f$ defined on the complex upper half-plane $\mathfrak{H}$.
If $f$ satisfies the usual transformation behaviour
\[
f(\frac{a\tau+b}{c\tau+d})=(c\tau+d)^kf(\tau)
\]
without the holomorphic/meromorphic assumption at $\infty$, $f$ defines a holomorphic function on the punctured disk
\[
\tilde{f}(q)=f(\tau), \enskip q=e^{2\pi i\tau}\in\mathbb{D}\setminus\{0\}.
\]
Therefore, $\tilde{f}$ admits a Laurent series expansion at $0$, with possibly infinite tale(=negative degree part).
To say $f$ is meromorphic at $0$ is to say the tail is finite, and holomorphic is to say it has no tale.

Now define for a pair of elliptic curve and an invariant differential $(E,\omega)$
\[
F(E,\omega)=(\omega_2)^{-k}f(\frac{\omega_1}{\omega_2})
\]
where Im$\frac{\omega_1}{\omega_2}>0$ and $\omega_1,\omega_2$ generate the periods $\{\int_{\gamma}\omega:\gamma\in$$H_1(E;\mathbb{Z})\}$.
Then we are asking about whether the function
\[
F(\mathbb{C}^{\times}/q^{\mathbb{Z}},2\pi izdz)
\]
is contained in $\mathbb{C}((q))$ or $\mathbb{C}[[q]]$.
Now, using the Weierstrass $\mathcal{P}$-functions and an appropriate affine coordinate change, the term appearing can easily be seen to be embedded in $\mathbb{P}^2$ as
\[
y^2+xy=x^3+B(q)x+C(q), \omega=\frac{dx}{x+2y}, \textnormal{ where}\enskip B(q), C(q) \in q\mathbb{Z}[[q]].
\]
This last equation defines an elliptic curve over $\mathbb{Z}((q))$.
We denote this by $Tate(q)$ and call it the $\textit{Tate Curve}$ and denote by $\omega_{\textnormal{can}}$ the differential form above.

Over this curve we have the Eisenstein series $E_4,E_6$ and the discriminant $\Delta$, and also the $j$-invariant
\[
j=\frac{1}{q}+744+\cdots, \enskip E_4=1+240\sum_{n\geq1}\sigma_3(n)q^n, \enskip E_6=1-504\sum_{n\geq1}\sigma_5(n)q^n, \enskip \Delta=q\prod_{n=1}^{\infty}(1-q^n)^{24}.
\]




\section{Coarse Moduli Scheme near $\infty$ via Tate Curve}


The formula from the last section shows that $A[[q]]=A[[1/j]]$.
The Tate curve from the last section looks like they might be useful for cusp analysis from the motivative definition above.
Indeed,

\begin{prp}
Let $A$ be a ring, $\mathcal{P}$ be a representable moduli problem on $Ell/A$, $\mathfrak{M}(\mathcal{P})$ the base scheme of the representing elliptic curve.
Then there exists a morphism
\[
\mathcal{P}_{Tate(q)/A((q))} \to \mathfrak{M}(\mathcal{P})_{A((q))}
\]
of $A((q))$ scheme which is a finite Galois covering, hence giving as isomorphism
\[
\mathcal{P}_{Tate(q)/A((q))}/\pm1 \simeq \mathfrak{M}(\mathcal{P})_{A((q))}
\]
where the $\pm1$ denotes the automorphism of the Tate curve $x \mapsto x$, $y \mapsto -x-y$.
\end{prp}

Recall the notations from the last notes:
$A$ is an excellent regular noetherian ring, $\mathcal{P}$ a finite relatively representable moduli problem on $Ell/A$, normal near infinity.
Then we obtain:

\begin{thm}
In the above settings, if some prime number $l$ is invertible in $A$ or $\mathcal{P}$ is representable near infinity.
Then the finite $A[[q]]$-scheme $\widehat{Cusps}(\mathcal{P})$ is the normalization of $A[[q]]$ in the finite normal $A((q))$ scheme
\[
\mathcal{P}_{Tate(q)/A((q))}/\pm1.
\]

\end{thm}

Therefore, to study the cusps one can study the Tate curve.
Moreover, if we introduce the commutative group scheme $T[N]$ over $\mathbb{Z}[q,q^{-1}]$ defined by
\[
T[N](A) =\{\textnormal{ collection of pairs }(X,i) \textnormal{ where }X\in A, \enskip 0 \leq i <N, \enskip X^N=q^i\} 
\]
with obvious multiplication $(X,i)*(Y,j)=(XY,i+j)$, then we can furthermore reduce the calculation to $T[N]$ for our `level $N$' moduli problems.

This seems plausible, because the groups $T[N]$ algother look like they are made to count torsion points on the Tate curve, and hence for `level $N$ problems', there must play a crucial role.
$T[N]$also possesses an alternating pairing $e_N:T[N] \times T[N] \to \mu_N$ defined by
\[
e_N(X,Y)=X^j/Y^i
\]
where $\mu_N$ is the group scheme
\[
\mu_N(A)=\{\textnormal{primitive } N^{\textnormal{th}} \textnormal{ root of unity of } A\}.
\]

Also from transcendental theory, one can recover the $x$ and $y$ of the Tate curve of $N$-torsion points as as elements of $\mathbb{Z}((q)) \otimes \mathbb{Z}[\zeta_N,1/N]$ where $\zeta_N$ is a primitive $N^{\textnormal{th}}$ root of unity.
Therefore, our level $N$ moduli problems should actually be considered at least over the ring $\mathbb{Z}[\zeta_N]$.

We will do that next week.
I will finish the notes for the course by stating theorems of the book concerning the cusps, and do some calculations for special case in the next week.











\end{document}























































