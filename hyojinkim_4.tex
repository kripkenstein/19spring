\documentclass[11pt,a4paper,reqno]{amsart} 
\usepackage{amsmath,amscd,amssymb,latexsym} 
\usepackage{tikz-cd}
\usepackage{longtable}

\numberwithin{equation}{section} 

\def\baselinestretch{1.2} 

\newtheorem*{main}{Main Theorem} 
\newtheorem{thm}{Theorem}[section] 
\newtheorem{lem}[thm]{Lemma} 
\newtheorem{prop}[thm]{Proposition} 
\newtheorem{defn}[thm]{Definition} 
\newtheorem{cor}[thm]{Corollary} 
\newtheorem{ex}[thm]{Example} 
\newtheorem{rmk}[thm]{Remark} 

\pagestyle{plain} 

\newenvironment{pf}{{\noindent \bf Proof:\ }}{\hfill $\Box$ \bigskip} 
\newenvironment{skpf}{{\noindent \bf Sketch of Proof:\ }}{\hfill $\Box$ \bigskip} 
\renewcommand{\baselinestretch}{1.10} 

\textwidth=17cm \textheight=25cm 

\addtolength{\topmargin}{-40pt} \addtolength{\oddsidemargin}{-2cm} 
\addtolength{\evensidemargin}{-2cm} 

\setlength{\unitlength}{1mm} 



\begin{document} 


\title{Homework 4 : The Finiteness of the Class Number} 
\author{Hyojin Kim} 
\date{May 3, 2019} 
\maketitle 

In this note and the following one, we will sketch the proof of that the class number of a number field is finite. 
 The method of proof gives an algorithm for computing the class group. 

\section{Basic notions} 
We will define some basic notions and prove briefly some propositions in this section.\\

Let $A$ be a Dedekind domain with field of fractions $K$,
 and let $B$ the integral closure of $A$ in a finite separable extension $L$.
 We want to define a homomorphism ${\rm Nm}:{\rm Id}(B)\rightarrow {\rm Id}(A)$
 which is compatible with taking norms of elements, i.e., such that the following diagram commutes:\
\begin{equation}\label{diagram}
\begin{tikzcd}
L^{\times} \arrow[r, "b \mapsto (b)"] \arrow[d, "{\rm Nm}"] 
& {\rm Id}(B) \arrow[d, "{\rm Nm}"] \\
K^{\times} \arrow[r, "a \mapsto (a)"]
& {\rm Id}(A).
\end{tikzcd}
\end{equation}

Let $\mathfrak{p}$ be a prime ideal of $A$,
 and factor $\mathfrak{p}B=\prod{{\mathfrak{B}_i}^{e_i}}$
 where $\mathfrak{B}_i$'s are the prime ideals dividing $\mathfrak{p}$
 and $e_i$'s are the ramification indices.
 If $\mathfrak{p}$ is principal, say $\mathfrak{p}=(\pi)$,
 then we should have
\[
{\rm Nm}(\mathfrak{p}B)={\rm Nm}(\pi \cdot B)={\rm Nm}(\pi)\cdot A = (\pi^m)=\mathfrak{p}^m,~m=[L:K].
\]
Also, because ${\rm Nm}$ is to be a homomorphism, we should have
\[
{\rm Nm}(\mathfrak{p}B)={\rm Nm}(\prod{{\mathfrak{B}_i}^{e_i}})=\prod {\rm Nm}({\mathfrak{B}_i})^{e_i}.
\]
On comparing these two formulas, we should define ${\rm Nm}(\mathfrak{B})=\mathfrak{p}^{f(\mathfrak{B}/\mathfrak{p})}$
 where $\mathfrak{p}=\mathfrak{B} \cap A$ and
 $f(\mathfrak{B}/\mathfrak{p})=[B/\mathfrak{B} : A/\mathfrak{p}]$. I sometimes use $\mathcal{N}$ to denote norms of ideals.

\begin{defn}
Let $\mathfrak{a}$ be a nonzero ideal in the ring of integers $\mathcal{O}_K$ of a number field $K$.
 Then $\mathfrak{a}$ is of finite index in $\mathcal{O}_K$, and we let $\mathbb{N}\mathfrak{a}$,
 the \textbf{numerical norm} of $\mathfrak{a}$, be this index:
\[
\mathbb{N}\mathfrak{a}=(\mathcal{O}_K : \mathfrak{a}).
\]
\end{defn}

\begin{rmk}
Let $\mathcal{O}_K$ be the ring of integers in a number field $K$.\\
(a) For any ideal $\mathfrak{a}$ in $\mathcal{O}_K$, ${\mathcal{N}_{K/\mathbb{Q}}}(\mathfrak{a})=(\mathbb{N}(\mathfrak{a}))$;
 therefore $\mathbb{N}(\mathfrak{a} \mathfrak{b} )=\mathbb{N}(\mathfrak{a})\mathbb{N}(\mathfrak{b})$.\\
(b) Let $\mathfrak{b} \subset \mathfrak{a}$ be fractional ideals in $K$; then 
\[
(\mathfrak{a} : \mathfrak{b} )=\mathbb{N}(\mathfrak{a}^{-1} \mathfrak{b} ).
\]
\end{rmk}

\begin{defn}
Let $V$ be a vector space of dimension $n$ over $\mathbb{R}$.
 A \textbf{lattice} $\Lambda$ in $V$ is a subgroup if the form
\[
\Lambda = \mathbb{Z} e_a + \cdots + \mathbb{Z} e_r
\]
with $e_1, \dots , e_r$ linearly independent elements of $V$.
 Thus a lattice is the free alelian subfroup of $V$ generated by elements of $V$ that are linearly independent over $\mathbb{R}$.
 When $r-n$, the lattice is said to be \textbf{full}.
\end{defn}

The subgroup $\mathbb{Z} + \mathbb{Z}\sqrt{2}$ of $\mathbb{R}$ is a free abelian group of rank 2, but it is \textit{not} a lattice in $\mathbb{R}$.

\begin{defn}
A subgroup $\Lambda$ of $V$ is said to be \textbf{discrete}
 if it is discrete in the induced topology.
 A topological space is discrete if its points (hence all subsets) are open,
 and so to say that $\Lambda$ is discrete means
 that every point $\alpha$ of $\Lambda$ has a neighbourhood $U$ in $V$ such that $U \cap \Lambda = \{ \alpha \}$.
\end{defn}

\begin{prop}
A subgroup $\Lambda$ of $V$ is a lattice if and only if it is discrete.
\end{prop}

It suffices to show that a discrete subgroup is a lattice and we shall argue by inducion on the order of a maximal $\mathbb{R}$-linearly independent subset of $\Lambda$.

\section{Finiteness of the class number} 

We will introduce the statements only and complete this section in the next note. \\

Let $K$ be an extension of degree $n$ of $\mathbb{Q}$, and let $\Delta _K$ be the discriminant of $K/\mathbb{Q}$.
 Let $2s$ be the number of nonreal complex embeddings of $K$.
 Then $B_K = \frac{n!}{n^n} \left(\frac{4}{\pi} \right)^s \mid \Delta _K \mid ^\frac{1}{2}$ is the \textbf{Minkowski bound}
 and the term $C_K = \frac{n!}{n^n} \left(\frac{4}{\pi} \right)^s$ is called the \textbf{Minkowski constant}.
 The last part of this section, we will show that $\mathbb{N}(\mathfrak{a}) \le B_K$.

\begin{thm} \label{4.4}
The class number of $K$ is finite.
\end{thm}

Let $K$ be a number field of degree $n$ over $\mathbb{Q}$. Suppose that $K$ has $r$ real embeddings $\{ \sigma_1,\dots, \sigma_r \}$ and $2s$ complex embedding $\{ \sigma_{r+1},\overline{\sigma_{r+1}},\dots, \sigma_{r+s},\overline{\sigma_{r+s}}\}$. Thus $n=r+2s$. We have an embedding
\[
\sigma : K \hookrightarrow {\mathbb{R}^r} \times {\mathbb{C}^s},~ \alpha \mapsto ( \sigma_1 \alpha, \dots, \sigma_{r+s} \alpha).
\]
We identify $V := {\mathbb{R}^r} \times {\mathbb{C}^s}$ with $\mathbb{R}^n$ using the basis $\{1,i\}$ for $\mathbb{C}$.

\begin{prop} 
Let $\mathfrak{a}$ be a nonzero ideal in $\mathcal{O}_K$; then $\sigma(\mathfrak{a})$ is a full lattice in $V$, and the volume of a fundamental parallelopiped of $\sigma(\mathfrak{a})$ is ${2^{-s}}\cdot{\mathbb{N}\mathfrak{a}}\cdot{\mid \Delta_K \mid}^\frac{1}{2}$.
\end{prop}

\begin{prop} 
Let $\mathfrak{a}$ be a nonzero ideal in $\mathcal{O}_K$.
Then $\mathfrak{a}$ contains a nonzero element $\alpha$ of $K$ with
\[
{\rm Nm}(\mathfrak{a}) \le B_K \cdot \mathbb{N}\mathfrak{a} = C_K \mathbb{N} \mathfrak{a} {\mid \Delta \mid}^{\frac{1}{2}}.
\]
\end{prop}

\begin{thm} \label{4.3}
Let $K$ be an extension of degree $n$ of $\mathbb{Q}$, and let $\Delta _K$ be the discriminant of $K/\mathbb{Q}$.
 Let $2s$ be the number of nonreal complex embeddings of $K$.
 Then there exists a set of representatives for the ideal class group of $K$ consisting of integral ideals $\mathfrak{a}$ with
\[
\mathbb{N}(\mathfrak{a}) \le \frac{n!}{n^n} \left(\frac{4}{\pi} \right)^s \mid \Delta _K \mid ^\frac{1}{2}.
\]
\end{thm}

\begin{thebibliography}{alpha} 
\bibitem{M} James. S. Milne, \emph{Algebraic Number Theory (v3.07)}, 2017. Available at www.jmilne.org/math/. 
\end{thebibliography} 

\end{document} 