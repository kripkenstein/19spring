%%Dohyeong: try to include an introduction
%%Dohyeong: try to avoid starting an article with `notation'

\documentclass[12pt,reqno]{amsart}
\usepackage[utf8x]{inputenc}
\usepackage{amssymb,amsmath,latexsym}
\usepackage{epsfig}

\usepackage{color}
\newcommand{\pk}{\color{magenta}}
\newcommand{\bk}{\color{black}}
%\usepackage{showkeys}
\renewcommand{\baselinestretch}{1.1}
\usepackage{epic,eepic,wrapfig,color, ifthen}
\usepackage{url}
 \oddsidemargin 0in
 \evensidemargin 0in
 \topmargin -0.6in
 \textwidth 6.5in
 \textheight 9in

\def\pf{{\medskip\noindent {\bf Proof. }}}
\def\qed{{\hfill $\Box$ \bigskip}}


\renewcommand{\atop}[2]{\substack{{#1}\\{#2}}}
\newcommand{\D}{\displaystyle}
\newcommand{\bP}{\mathbb{P}}
\newcommand{\PP}{\mathbb{P}}
\newcommand{\cP}{{\mathcal{P}}}
\newcommand{\cK}{{\mathcal{K}}}
\newcommand{\ZE}{{E_{\mathbb Z}\times E}}
\newcommand{\cA}{{\mathcal{A}}}
\newcommand{\cD}{{\mathcal{D}}}
\newcommand{\bE}{{\mathbb{E}}}
\newcommand{\bR}{{\mathbb{R}}}
\def\eps{\varepsilon}
\newcommand{\af}{\alpha}
\newcommand{\bt}{\beta}
\newcommand{\gm}{\gamma}
\newcommand{\dt}{\delta}
\newcommand{\ep}{\varepsilon}
\newcommand{\zt}{\zeta}
\newcommand{\te}{\theta}
\newcommand{\ld}{\lambda}
\newcommand{\sm}{\sigma}
\newcommand{\kp}{\kappa}
\newcommand{\rk}{\rm rank}
\newcommand{\wt}{\widetilde}
\newcommand{\sms}{\setminus}
\newcommand{\ift}{\infty}
\newcommand{\ds}{\displaystyle}
\newcommand{\Gm}{\Gamma}
\newcommand{\Ld}{\Lambda}
\newcommand{\vp}{\varphi}
\def\1{{\bf 1}}


\def\sA {{\cal A}} \def\sB {{\cal B}} \def\sC {{\cal C}}
\def\sD {{\cal D}} \def\sE {{\cal E}} \def\sF {{\cal F}}
\def\sG {{\cal G}} \def\sH {{\cal H}} \def\sI {{\cal I}}
\def\sJ {{\cal J}} \def\sK {{\cal K}} \def\sL {{\cal L}}
\def\sM {{\cal M}} \def\sN {{\cal N}} \def\sO {{\cal O}}
\def\sP {{\cal P}} \def\sQ {{\cal Q}} \def\sR {{\cal R}}
\def\sS {{\cal S}} \def\sT {{\cal T}} \def\sU {{\cal U}}
\def\sV {{\cal V}} \def\sW {{\cal W}} \def\sX {{\cal X}}
\def\sY {{\cal Y}} \def\sZ {{\cal Z}}

\def\bA {{\mathbb A}} \def\bB {{\mathbb B}} \def\bC {{\mathbb C}}
\def\bD {{\mathbb D}} \def\bE {{\mathbb E}} \def\bF {{\mathbb F}}
\def\bG {{\mathbb G}} \def\bH {{\mathbb H}} \def\bI {{\mathbb I}}
\def\bJ {{\mathbb J}} \def\bK {{\mathbb K}} \def\bL {{\mathbb L}}
\def\bM {{\mathbb M}} \def\bN {{\mathbb N}} \def\bO {{\mathbb O}}
\def\bP {{\mathbb P}} \def\bQ {{\mathbb Q}} \def\bR {{\mathbb R}}
\def\bS {{\mathbb S}} \def\bT {{\mathbb T}} \def\bU {{\mathbb U}}
\def\bV {{\mathbb V}} \def\bW {{\mathbb W}} \def\bX {{\mathbb X}}
\def\bY {{\mathbb Y}} \def\bZ {{\mathbb Z}}
\def\R {{\mathbb R}} \def\RR {{\mathbb R}}
\def\n{{\bf n}}
\def\loc{{\rm loc}}


\def\BB{{\cal B}}
\def\wh{\widehat}
\def\wt{\widetilde}
\def\pf{\noindent{\bf Proof.} }
\def\beq{\begin{equation}}
\def\eeq{\end{equation}}
\def\bee{\begin{equation}}
\def\eee{\end{equation}}
\def\ep{\epsilon}
\def\de{\delta}
\def\til{\tilde}
\def\nn{\nonumber}

\newcommand{\expr}[1]{\left( #1 \right)}
\newcommand{\cl}[1]{\overline{#1}}
\newtheorem{thm}{Theorem}[section]
\newtheorem{lemma}[thm]{Lemma}
\newtheorem{defn}[thm]{Definition}
\newtheorem{definition}[thm]{Definition}
\newtheorem{prop}[thm]{Proposition}
\newtheorem{proposition}[thm]{Proposition}
\newtheorem{corollary}[thm]{Corollary}
\newtheorem{cor}[thm]{Corollary}
\newtheorem{remark}[thm]{Remark}
\newtheorem{example}[thm]{Example}
\numberwithin{equation}{section}
\def\ee{\varepsilon}
\def\qed{{\hfill $\Box$ \bigskip}}
\def\NN{{\cal N}}
\def\AA{{\cal A}}
\def\MM{{\cal M}}
\def\BB{{\cal B}}
\def\CC{{\cal C}}
\def\LL{{\cal L}}
\def\DD{{\cal D}}
\def\FF{{\cal F}}
\def\EE{{\cal E}}
\def\QQ{{\cal Q}}
\def\RR{{\mathbb R}}
\def\R{{\mathbb R}}
\def\L{{\bf L}}
\def\K{{\bf K}}
\def\S{{\bf S}}
\def\A{{\bf A}}
\def\Z{{\mathbb Z}}
\def\Q{{\mathbb Q}}
\def\E{{\mathbb E}}
\def\F{{\mathbb F}}
\def\P{{\mathbb P}}
\def\N{{\mathbb N}}
\def\eps{\varepsilon}
\def\wh{\widehat}
\def\wt{\widetilde}
\def\pf{\noindent{\bf Proof.} }
\def\beq{\begin{equation}}
\def\eeq{\end{equation}}
\def\bee{\begin{equation}}
\def\eee{\end{equation}}
\def\rd{\mathrm d}
\def\Dc{{\overline D}^c}
\def\Uc{{\overline U}^c}
\def\Rd{\bR^d}
\def\sms{\setminus}



\begin{document}


\title
{ \bf Review: A Heuistic for Boundedness of Ranks of Elliptic Curves}
 \author{Soobin Cho}

\maketitle

{\bf Notation.} For $x=(x_1, ..., x_m)$ and $a=(a_1, ..., a_n)$, the notation $f(x,a) \ll_a g(x,a)$ means that for every fixed $a$, there exists a positive constant $C(a)$ such that $f(x,a) \le C(a) g(x,a)$ for all $x$. Then, $f(x,a) \asymp_a g(x,a)$ means that $f(x,a) \ll_a g(x,a)$ and $g(x,a) \ll_a f(x,a)$.

For an abelian group $G$ and $n \in \N$, denote by $G[n]:= \{x \in G : nx = 0 \}$. For $p$ prime, define $G[p^\infty]:= \cup_{m \in \N} G[p^m]$ and define the $p$-rank of $G$ to be ${\rm dim_{\F_p}} G[p]$.

For a commutative ring $R$, denote by $M_n(R)$ be the set of $n \times n$ matrices with entries in $R$. For $X >0$, let $M_n(\Z)_{\le X} \subset M_n(\Z)$ be the subset of matrices whose entries have absolute value less than or equal to $X$. We also let $M_n(R)_{\rm alt}$ be the set of alternating matrices, i.e. $A^T = -A$ and all the diagonal entries are $0$.

For a subset $S \subset M_n(\Z_p)$, define ${\rm Prob}(S) = {\rm Prob}(S | A \in M_n(\Z_p))$ as the probability of $S$ with respect to the normalized Haar measure on the compact group $M_n(\Z_p)$.


\section{Introduction and History}

It is well known that the set $E(\Q)$ of rational points of an elliptic curve $E$ over $\Q$ has the structure of an abelian group. In 1922, Mordell proved that ${\rm rk \;} E(\Q) < \infty$. Then, it is natural to ask the question of boundedness: 

\smallskip
{\bf Conjecture $\mathbf{1}$.} Does there exists a constant $B>0$  such that for every elliptic curve $E$ over $\Q$, one has ${\rm rk }\; E(\Q) \le B$?

\smallskip

In the article, the authors presented a probabilistic model providing a heuristic for the arithmetic of elliptic curves and proved theorems about the model that suggest  ${\rm rk \;} E(\Q) \le 21$ for all but finitely many elliptic curves $E$.

\subsection{Brief history of boundedness guesses.}

Many authors have proposed guesses as to whether Conjecture $1$ is true, and their thoughts have shifted from positive to negative over time. In 1960, Honda conjectured that even for any abelian variety $A$ over $\Q$, there is a constant $c_A$ such that ${\rm rk }\; A(K) \le c_A[K:\Q]$ for every number field $K$ not only when $K = \Q$. However, from the mid-1960s to the present, it seems that the common belief is that ranks are unbounded. Here are two possible reasons for this opinion shift towards unboundedness:

\smallskip
 1. Tate and Shafarevich (1967) and Ulmer (2002) constructed families of elliptic curves over $\F_p(t)$ (not a number field) in which the rank is unbounded.

 2. The lower bound for the maximum rank of an elliptic curve over $\Q$ has been increasing. The current record is held by Elkies (2006), who found an elliptic curve $E$ over $\Q$ of rank $\ge 28$, and an infinite family of elliptic curves over $\Q$ of rank $\ge 19$.

\smallskip

Some authors have even proposed a rate at which rank grow relative to the conductor $N$:

\indent $\bullet$ Ulmer (2002),
$$
\limsup_{N \to \infty} \frac{ {\rm rk }\; E(\Q) } { \log N/ \log \log N} > 0?
$$
\indent $\bullet$ Farmer, Gonek and Hughes (2007),
$$
\limsup_{N \to \infty} \frac{ {\rm rk }\; E(\Q) } { \sqrt{\log N \log \log N}} = 1?
$$

\vspace{2mm}

\subsection{Conjectures for rank $2$ asymptotics.}

We first recall some basic notions in the theory of elliptic curves.

\begin{definition}
{\rm
(1) (Quadratic twist) First assume that $char(K) \neq 2$. Let $E$ be an alliptic curve over $K$ of the form:
\begin{align*}
y^2 = x^3 + a_2x^2 + a_4x + a_6.
\end{align*}
Given $d \in K\setminus K^2$, the quadratic twist of $E$ is the curve $E_d$, defined by the equation:
\begin{align*}
dy^2 = x^3 + a_2x^2 + a_4x + a_6.
\end{align*}
Observe that $E_d(x,y) = 0$ if and only if $E(x,y\sqrt{d}) = 0$. Hence, the two elliptic curves $E$ and $E_d$ are isomorphic over the field extension $K(\sqrt{d}) \cong K[X]/(X^2-d)$.

Now assume that $char(K) = 2$. Let $E$ be an elliptic curve over $K$ of the form:
\begin{align*}
y^2 + a_1xy + a_3y = x^3 + a_2x^2 + a_4x + a_6.
\end{align*}
Given $d \in K\setminus \{0\}$, the quadratic twist of $E$ is the curve $E_d$, defined by the equation:
\begin{align*}
y^2 + a_1xy + a_3y = x^3 + (a_2 + da_1^2)x^2 + a_4x + a_6 + da_3^2.
\end{align*}
In this case, we can check that $E_d(x,y) = 0$ if and only if $E(x, y+(a_1x+a_3)\zeta) =0$ where $\zeta$ is any of the solutions of the equation $X^2+X+d=0$ in fixed algebraic closure of $K$. Hence, the two elliptic curves $E$ and $E_d$ are isomorphic over the field extention $K[X]/(X^2+X+d)$.

\smallskip

(2) (Fundamental discriminant) $D \in \Z$ is a fundamental discriminant if and only if one of the following statements holds:
\begin{align*}
&\indent \bullet D \equiv 1 \; \text{(mod 4)} \; \text{and is square-free}; \\
&\indent \bullet D = 4m, \; \text{where} \; m \equiv 2 \; \text{or} \; 3 \; \text{(mod 4) and } m \; \text{is square-free}.
\end{align*}

There exists a one-to-one correspondence between the set of fundamental discriminants with the union of set of quadratic fields and $\Q$, that is, each nontrivial fundamental discriminant is the discriminant of a unique (up to isomorphism) quadratic number field.

\smallskip

(3) ((naive) Height) An elliptic curve $E$ over $\Q$ is isomorphic to the projective closure of a curve $y^2 = x^3 + Ax + B$ for a unique pair of integers $(A,B)$ such that there is no prime $p$ such that $p^4 | A$ and $p^6 | B$. Define the (naive) height of $E$ by
\begin{align*}
\text{ht} \; E:= \max \{|4A^3|, |27B^2|\}.
\end{align*}

\smallskip

(4) (Conductor for the simplified form) Let an elliptic curve $E$ over $\Q$ has a Weierstrass equation in the simplified form $y^2 = x^3 + Ax + B$. Let $p$ be a prime in $\Z$. By reducing each of the coefficients $A$ and $B$ modulo $p$, we obtain the equation of a cubic curve $\widehat{E}$ over the finite field $\mathbb{F}_p$. If $\widehat{E}$ is a non-singular curve, then we say that $E$ has good reduction at $p$. Else if $\widehat{E}$ has a cusp (i.e. the discriminant of $\widehat{E}$ equals to $0$ and $A = 0$ (mod $p$)), then we say that $E$ has additive reduction at $p$. Otherwise, if $\widehat{E}$ has a node,  (i.e. the discriminant of $\widehat{E}$ equals to $0$ and $A \neq 0$ (mod $p$)), then we say that $E$ has multiplicative reduction at $p$.

For each prime $p \in \Z$, define the quantity $f_p$ as follows:
\begin{align*}
f_p = \begin{cases}
0, & \mbox{if} \; E \; \text{has good reduction at} \; p, \\
1, & \mbox{if} \; E \; \text{has multiplicative reduction at} \; p, \\
2, & \mbox{if} \; E \; \text{has additive reduction at} \; p, \; \text{and} \; p \neq 2,3, \\
2+ \delta_p, & \mbox{if} \; E \; \text{has additive reduction at} \; p, \; \text{and} \; p \in \{2,3\}.
\end{cases}
\end{align*}

Then, the conductor $N_{E/\Q}$ of an elliptic curve $E$ over $\Q$ is defined as
\begin{align*}
N_{E/\Q} := \prod_{p: \; \text{prime}} p^{f_p}.
\end{align*}

}
\end{definition}

\begin{example}\label{htEd}
{\rm
 Let $E$ be an alliptic curve over $\Q$ of the form $y^2 = x^3 + Ax +B$ for some constants $A$ and $B$ such that $4A^3 + 27B^2 \neq 0$. Then, for each $d \in \Q \setminus \Q^2$, the quadratic twist of $E_d$ is defined by the equation $dy^2 = x^3 + Ax + B$. We can check that this is equivalent to the equation $y^2 = x^3 + d^2Ax + d^3B$. Hence, we obtain $\text{ht} \; E_d = \max \{ |4d^6A^3|, |27d^6B^2|\} \asymp d^6$ for general elliptic curve $E$ over $\Q$.
}
\end{example}
\vspace{2mm}

Fix an elliptic curve $E$ over $\Q$. Let $d$ range over fundamental discriminants in $\Z$. Given $r \in Z_{\ge 0}$ and $D>0$, define
\begin{align*}
N_{\ge r}(D)&:= \# \{d: |d| \le D, \; \text{rk} \; E_d(\Q) \ge r \},\\
N_{\ge r, \; \text{even}}(D)&:= \# \{d: |d| \le D, \; \text{rk} \; E_d(\Q) \ge r, \;\text{and} \; w(E_d) = +1 \}, \\
N_{\ge r, \; \text{odd}}(D)&:= \# \{d: |d| \le D, \; \text{rk} \; E_d(\Q) \ge r, \;\text{and} \; w(E_d) = -1 \},
\end{align*}
where $w(E_d) \in \{-1, +1\}$ is the global root number of $E_d$.

\smallskip
{\bf Conjecture $\mathbf{2}$.} Does it hold that
\begin{align*}
N_{\ge 2, \; \text{even}}(D) = D^{3/4 +o(1)}\; ?
\end{align*}

\smallskip

In other words, the prediction is that for $d$ such that $w(E_d) = +1$, the probability that $\text{rk} \; E_d(\Q) \ge 2$ should be about $d^{3/4+o(1)}/d \simeq d^{-1/4}$. Since $\text{ht} \; E_d \asymp d^6$ by Example \ref{htEd}, this prediction corresponds to a probability of $h^{-1/24}$ for an elliptic curve of height $h$.

\smallskip

\begin{remark}
{\rm
(a) The Birch and Swinnerton-Dyer conjecture would imply the parity conjecture,

\smallskip

{\bf Conjecture $\mathbf{3}$.} Does it hold that
\begin{align*}
w(E) = (-1)^{\text{rk} \; E(\Q)} \; ?
\end{align*}

\smallskip

Let $E$ be an elliptic curve over $\Q$ with $w(E) = +1$. Then, it is known that for a weight $3/2$ cusp form $f = \sum a_nq^n$ such that for all odd fundamental discriminants $d<0$ coprime to the conductor of $E$, we have $a_{|d|} = 0$ if and only if $L(E_d, 1) = 0$. If the BSD conjecture is true, then the condition $L(E_d,1)=0$ is equivalent to $\text{ord}_{s=1} L(E_d, s) \ge 2$, which is equivalent to $\text{rk} \; E_d(\Q) \ge 2$. The Ramanujan conjecture predicts that $a_|d|$ is an integer satisfying $|a_|d|| \le |d|^{1/4 + o(1)}$. Hence, heuristically, we can expect that $a_|d| = 0$ occurs with "probability" $|d|^{-1/4+o(1)}$ and hence $N_{\ge 2, \; \text{even}}(D)  \simeq \sum_{|d| \le D} |d|^{-1/4+o(1)} \simeq |D|^{3/4 + o(1)}$.

\smallskip

(b) Conrey, Keating, Rubinstein and Snaith used random matrix theory to get a developed conjecture, that is, there exist constants $c_E, e_E \in \R$ such that

\smallskip
{\bf Conjecture $\mathbf{4}$.}
\begin{align*}
N_{\ge 2, \; \text{even}}(D) = (c_E + o(1))D^{3/4} (\log D)^{e_E}\; ?
\end{align*}

On the other hand, Watkins developed a variant for the family of all elliptic curves over $\Q$, that is, there exists a constant $c_0>0$ such that

\smallskip
{\bf Conjecture $\mathbf{5}$.}
\begin{align*}
\# \{E: \text{ht} \; E \le H, \; \text{rk} \; E_d(\Q) \ge 2, \;\text{and} \; w(E_d) = +1 \} =  (c_0 + o(1)) H^{19/24} (\log H)^{3/8}\; ?
\end{align*}

\smallskip

An elementary seive argument shows that 
\begin{align*}
\# \{E: \text{ht} \; E \le H\} = (\kappa + o(1))H^{5/6},
\end{align*}
where $\kappa:=2^{4/3}3^{-3/2} \zeta(10)^{-1}$. Hence, the Conjecture $5$ is related to Conjecture $4$ through the equation that $19/24 = 5/6 - 1/24$.

}
\end{remark}

\end{document}
