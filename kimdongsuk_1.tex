\documentclass[11pt]{amsart}
\usepackage{amssymb}
\usepackage{amsmath}
\usepackage{amscd}
\usepackage{amsfonts}
\usepackage{amssymb,amscd,amsthm,latexsym,verbatim}
\usepackage{amssymb,amscd,amsthm,verbatim,syntonly}
\usepackage{graphics}
\usepackage{latexsym}
\usepackage{mathrsfs}
\usepackage{enumitem}
\usepackage{cite}
\usepackage{graphicx,subfigure}
\usepackage{blkarray}
\usepackage[left=3.6cm,right=3.6cm,top=3cm,bottom=3cm]{geometry}
\usepackage{indentfirst}
\parindent=1em

\newtheorem{thm}{Theorem}[section]
\newtheorem{f}{Fact}
\newtheorem{theom}{Theorem}
\newtheorem{claim}{claim}
\newtheorem{lem}[thm]{Lemma}
\newtheorem{cor}[thm]{Corollary}
\newtheorem{conj}[thm]{Conjecture}
\newtheorem{prop}[thm]{Proposition}
\newtheorem{rmk}[thm]{Remark}
\newtheorem{qe}[thm]{Question}
\renewcommand{\theequation}{\thesection.\arabic{equation}}
\theoremstyle{definition}
\newtheorem{exmp}[thm]{Example} 
\newtheorem{defn}[thm]{Definition}
\newtheorem{q}{Question}


\newcommand{\redhom}{\tilde{H}}
\newcommand{\Aut}{\operatorname{Aut}}
\newcommand{\sgn}{\operatorname{sgn}}

\begin{document}
\title[]
{Riemann Surfaces}
\author{Dong Suk Kim}


\maketitle
\section{Riemann surfaces}
In view of historical origin and current development in theoretical physics, Riemann surfaces are very important topics. A Riemann surface is a one dimensional complex manifold. But why we call the manifolds Riemann surfaces? Through 5 essay, we shall see the reason and connections with elliptic curves. 

\begin{defn} Let $X$ be a two-dimensional manifold. $A$ $complex$ $chart $ on X is a homeomorphism $ \phi : U \to V $, where $U \subset X$ is an open set in $X$, and $V \subset \mathbb{C}$ is an open set in the complex plane. Let $\phi_1 : U_1 \to V_1\ $ and $\phi_2 : U_2 \to V_2 $ be two complex charts on X. We say that $\phi_1$ and $\phi_2$ are compatible if  either $ U_1 \cap U_2 = \varnothing $, or 
\begin{equation*} \phi_2 \circ \phi_1^{-1} : \phi_1(U_1 \cap U_2) \to \phi_2(U_1 \cap U_2)
\end{equation*} is holomorphic.

\end{defn}
\begin{defn}A $complex$ $atlas$ on X is a system $\mathcal{U} = \{ \phi_i : U_i \to V_i, i \in I \}$ of charts which are compatible and cover X.
\end{defn}
\begin{defn} Two complex atalses $\mathcal{A}$ and $\mathcal{B}$ are $equivalent$ if every chart of one is compatible with every chart of the other.
\end{defn}

\begin{defn} A $complex$ $structure$ on X is a maximal complex atlas on X, or equivalently, and equivalence class of complex atlases on X
\end{defn}

\begin{defn} A $Riemann$ $surface$ is a connected two-dimensional manifold X with a complex structure.
\end{defn} 
\begin{exmp} $The$ $Riemann$ $sphere$ $\mathbb{P}^1$. $\mathbb{P}^1$ = $\mathbb{C}$ $\cup$ $\{\infty \}$ endowed with one point compactification of $\mathbb{C}$ and complex structure. One can identify with the unit sphere in $\mathbb{R}^{3}$ using the stereographic projection.
\end{exmp}

\begin{exmp} $Tori$. Suppose $\omega_1$, $\omega_2$ $\in$ $\mathbb{C}$ are linearly independent over $\mathbb{R}$. Define 
\begin{equation*} \Gamma := \mathbb{Z}\omega_1 + \mathbb{Z}\omega_2 = \{n\omega_1 + m\omega_2 : n,m \in \mathbb{Z} \}.
\end{equation*}  
The Quotient topology on $ \mathbb{C}/\Gamma $ endowed with a complex structure is a Tori. For detail, See [1].
\end{exmp}

\begin{defn}
Suppose X and Y are Riemann surfaces. A continuous mapping $f: X \to Y$ is called $holomorphic$, if for every pair of charts $\psi_1 : U_1 \to V_1$ on $X$ and $\psi_2 : U_2 \to V_2$ on $Y$ with $f(U_1) \subset U_2$, the mapping \begin{equation*} \psi_2 \circ f \circ \psi_1^{-1} : V_1 \to V_2
\end{equation*} is holomorphic in the usual sense.

A mapping $f: X \to Y$ is called $biholomorphic$ if it is bijective and both $f,f^{-1}$ are holmorphic. Two Riemann surface X and Y are called $isomorphic$ if there exists a biholomorphic mapping $f:X \to Y$.
\end{defn}
\begin{prop} Let $\Gamma = \mathbb{Z}\omega_1 + \mathbb{Z}\omega_2$ and $\Gamma' = \mathbb{Z}\omega_1' + \mathbb{Z}\omega_2'$ be two lattices in $\mathbb{C}$. Then $\Gamma = \Gamma'$ if and only if there exists a matrix $A \in SL(2,\mathbb{Z}) := \{A \in GL(2,\mathbb{Z}):det A=1 \}$ such that 
\begin{equation*} { \omega_1' \choose \omega_2'} = A { \omega_1 \choose \omega_2}.
\end{equation*}
\end{prop}

$Proof$. It suffices to show 'only if' part. If we let A = $\begin{pmatrix} a & b \\ c & d \end{pmatrix}, ad-bc=1$ since $A \in SL(2,\mathbb{Z})$. Then $c\omega_1' - a\omega_2' = (cb-ad)\omega_2=-\omega_2$. Then $\omega_2 \in \Gamma'$. Since $\omega_1' = a\omega_1+b\omega_2$ and $a,b$ are integers, also $\omega_1  \in \Gamma'$.  Hence $ \Gamma \subset \Gamma'$ . As $A^{-1}$ $\in SL(2,\mathbb{Z})$, $\Gamma' \subset \Gamma$ by same argument.

\vspace{2ex}

Next we want to propose more concrete criterion for classifying complex torus.
\begin{prop} Every torus $X$ = $\mathbb{C}/\Gamma$ is isomorphic to a torus of the form 
\begin{equation*} X(\tau):=\mathbb{C}/(\mathbb{Z}+\mathbb{Z}\tau),
\end{equation*}
where $\tau$ $\in \mathbb{C}$ satisfies $Im(\tau)>0$.
\end{prop}
$Proof$. Let $\omega_1, \omega_2$ be generator of $\Gamma$. Then we can choose $\alpha$ such that $\alpha\omega_1$ is real in complex plane and $\left\vert \alpha \right\vert$=$(\left\vert \omega_1 \right\vert)^{-1}$. If arg($\alpha\omega_2) > \pi$, replace by $-\alpha$.
If we set $\tau$=$\alpha\omega_2$, multiplying $\alpha$ on $\mathbb{C}$ induces a holomorphic map from $\mathbb{C}/\Gamma$ to $X(\tau)$ since $\alpha\Gamma \subset \Gamma'$. But the two lattice is identified by multiplying $\alpha$. So the map is biholomorphic. Hence the two tori is isomorphic. Note that the volume of the lattice of $X(\tau)$ is differ from that of $\Gamma$.



\begin{cor} Suppose $\begin{pmatrix} a & b \\ c & d \end{pmatrix}$ $\in SL(2,\mathbb{Z})$ and $Im(\tau)>0$. Let
\begin{equation*} \tau':={a\tau+b \over c\tau+d}.
\end{equation*}
Then the tori $X(\tau)$ and $X(\tau')$ are isomorphic.
\end{cor}
$Proof$.  Set ${\omega_1 \choose \omega_2 }=\begin{pmatrix} a & b \\ c & d \end{pmatrix}$${\tau \choose 1}$. Let $\Gamma'$,$\Gamma$ be each lattice generated by $(\tau',1)$ and $(\tau,1)$. Let $\alpha $=$(c\tau+d)^{-1}$. Then $\Gamma'$=$\alpha\Gamma$.
Then by proposition (1.9),(1.10), $X(\tau)$ and $X(\tau')$ are isomorphic.  
\vspace{2ex}

In general, Given a map from $\mathbb{C}/\Gamma$ to $\mathbb{C}/\Gamma'$, then there exists $\alpha,\beta$ with $\alpha\Gamma \subset \Gamma'$ such that the image of map is $\alpha\Gamma+\beta$+$\Gamma'$.  
 

\begin{thebibliography}{30}
\bibitem{OF}
O.Forster, \emph{Lectures on Riemann Surfaces}. Springer-Verlag, New York, 1999.


\end{thebibliography}

\end{document}



