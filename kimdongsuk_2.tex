\documentclass[11pt]{amsart}
\usepackage{amssymb}
\usepackage{amsmath}
\usepackage{amscd}
\usepackage{amsfonts}
\usepackage{amssymb,amscd,amsthm,latexsym,verbatim}
\usepackage{amssymb,amscd,amsthm,verbatim,syntonly}
\usepackage{graphics}
\usepackage{latexsym}
\usepackage{mathrsfs}
\usepackage{enumitem}
\usepackage{cite}
\usepackage{graphicx,subfigure}
\usepackage{blkarray}
\usepackage[left=3.6cm,right=3.6cm,top=3cm,bottom=3cm]{geometry}
\usepackage{indentfirst}
\parindent=1em

\newtheorem{thm}{Theorem}[section]
\newtheorem{f}{Fact}
\newtheorem{theom}{Theorem}
\newtheorem{claim}{claim}
\newtheorem{lem}[thm]{Lemma}
\newtheorem{cor}[thm]{Corollary}
\newtheorem{conj}[thm]{Conjecture}
\newtheorem{prop}[thm]{Proposition}
\newtheorem{rmk}[thm]{Remark}
\newtheorem{qe}[thm]{Question}
\renewcommand{\theequation}{\thesection.\arabic{equation}}
\theoremstyle{definition}
\newtheorem{exmp}[thm]{Example} 
\newtheorem{defn}[thm]{Definition}
\newtheorem{q}{Question}


\newcommand{\redhom}{\tilde{H}}
\newcommand{\Aut}{\operatorname{Aut}}
\newcommand{\sgn}{\operatorname{sgn}}

\begin{document}

\title[]
{Elliptic functions and Elliptic integrals}
\author{Dong Suk Kim}

\maketitle
\begin{abstract} 

 In this article, we briefly browse the history of elliptic functions and addition theorem. 

\end{abstract}


\section{Addition Theorem}

\begin{defn} An addtion theorem for a function $f$ is a formula expressing $f(u+v)$ in terms of $f(u)$ and $f(v)$.
\end{defn}

Here $f$ is an one variable function. 


\vspace{2ex}

\begin{exmp} For $F(x)$=$\int_{0}^{x} {1 \over \sqrt{1-u^{2}}}\, du$, 
\begin{equation*} \int_{0}^{\sin(x)} {1 \over \sqrt{1-u^{2}}}\, du + \int_{0}^{\sin(y)} {1 \over \sqrt{1-u^{2}}}\, du = \int_{0}^{\sin(x+y)} {1 \over \sqrt{1-u^{2}}}\, du.
\end{equation*}
 \end{exmp}

\vspace{2ex}

\begin{exmp}[Euler's addition theorem] For
$F(x)$=$\int_{0}^{x} {1 \over \sqrt{1-u^{4}}}\, du$, 
\begin{equation*} \int_{0}^{x} {1 \over \sqrt{1-u^{4}}}\, du + \int_{0}^{y} {1 \over \sqrt{1-u^{4}}}\, du = \int_{0}^{ {x\sqrt{1-y^{4}}+y\sqrt{1-x^{4}} \over 1+x^{2}y^{2}} } {1 \over \sqrt{1-u^{4}}}\, du.
\end{equation*} 
If we set $f(s)=x, f(t)=y$, then
\begin{equation*} \int_{0}^{f(s)} {1 \over \sqrt{1-u^{4}}}\, du + \int_{0}^{f(t)} {1 \over \sqrt{1-u^{4}}}\, du = \int_{0}^ {f(s+t)} {1 \over \sqrt{1-u^{4}}}\, du.
\end{equation*} 

Here $f$ is the inverse of $F$.
In short, an addition theorem is related to the inverse functions of the antiderivative of a function.

\end{exmp}

\vspace{2ex}
Note that the integral is an elliptic integral. Elliptic integral is $\int R(x,\sqrt{P(x)})\, dx$ where $R(x,y)$ is a rational function of two variables $x,y$ and $P(x)$ is a polynomial of degree 3 or 4. That such integrals cannot be evaluated in terms of the elementary functions was finally proved by Liouville.
\newline 
\indent Let 
\begin{equation*}P(x)=a_0x^{2n-1}+a_1x^{2n-2}+...+a_{2n-1}
\end{equation*}
set $x=c+1/y$, where $c$ is not a root of $P(x)$. Then
\begin{equation*} P(x)=P(c)+P'(c){1 \over y}+...+{P^{2n-1}(c) \over (2n-1)!}{1 \over y^{2n-1}}= {P_1(y) \over y^{2n}}.
\end{equation*} 
Then $P_1(x)$ has degree 2n. Thus $\int R(x,\sqrt{P(x)})\, dx$=$\int R_1(x,\sqrt{P_1(x)})\, dx$.
\newline

Legendre showed that integration of the elliptic integral of fourth degree, can be reduced to the integration of the three integrals,
\begin{equation*} \int {1 \over \sqrt{1-u^{2}}\sqrt{1-l^{2}u^{2}}}\, du, \int {u^{2} \over \sqrt{1-u^{2}}\sqrt{1-l^{2}u^{2}}}\, du, \int {1 \over (u-a)\sqrt{1-u^{2}}\sqrt{1-l^{2}u^{2}}}\, du.
\end{equation*}
\newline
Next, Weierstrass showed that 
\begin{equation*} u(p)= \int_{0}^{p} {1 \over \sqrt{4t^{3}-g_2t-g_3}}\, dt,
\end{equation*} where $g_2$ and $g_3$ are constants such that $g_2^{3}-27g_3^{2} \ne0$ is the fundamental elliptic function.
the fundamentality means every elliptic function could be expressed in terms of $p(u)$ and $p'(u)$. It also related to the inverse function of the function defined by antiderivative.
\newline
\indent Abel generalized an addition theorem.
\begin{thm} Let $\psi x$ =$\int {R(x,\sqrt{P(x)})}\, dx$. Let $n$ be positive integers, and $h_1,..,h_n$ rational numbers. Then there exist $g$ that
\begin{equation*}  h_1\psi x_1+..+h_n\psi x_n = v+\psi x'_1+..+\psi x'_g
\end{equation*} where $v$ is an elementary function of $x_1,..,x_n$ and $x'_1,..,x'_g$ are algebraic function of $x_1,..,x_n$ and $g$ is minimal : given algebraic functions $x'_1,..,x'_{g-1}$ of $x_1,..,x_g$, for any constant $v$, there exists $\psi x$ such that
\begin{equation*} \psi x_1+..+\psi x_g \ne \psi x'_1+..+\psi x'_{g-1}+v
\end{equation*}
Here $\psi$ is another bounded integral made by $\sqrt{P(x)}$.
\end{thm}

\indent Next time We shall discuss what $g$ means.

\begin{thebibliography}{30}
\bibitem{OF}
O.Forster, \emph{Lectures on Riemann Surfaces}. Springer-Verlag, New York, 1999.
\bibitem{OR}
Olav Afrnfinn Laudal, Ragni Piene \emph{The legacy of Niels Henrik Abel}. Springer-Verlag, New York, 2004
\bibitem{BJ}
Barrios, Jose \emph{A Brief History of Elliptic Integral Addition Theorems}. Ros-Hulman Undergraduate Mathematics Journal:Vol.10, 2009
\bibitem{KM}
Kline, Morris, \emph{Mathematical Thought from Ancient to Modern Times},vol. 2, Oxford University Press: New York, pages 421, 422, 646, 1990.
\end{thebibliography}

\end{document}