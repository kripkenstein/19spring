
\documentclass[a4paper,11pt]{article}


\usepackage{amsmath}
\usepackage{amssymb}
\usepackage{tikz-cd}
\usepackage{amsthm}
\usepackage{mathtools}
\usepackage{titling}
\usepackage[left=2cm,right=2cm,top=1cm,bottom=2cm,a4paper]{geometry}
\usepackage{mathrsfs}
\usepackage{calrsfs}



\begin{document}

\theoremstyle{plain}
\newtheorem{thm}{Theorem}[section]
\newtheorem{defn}[thm]{Definition}
\newtheorem{exm}[thm]{Example}
\newtheorem{prp}[thm]{Proposition}
\newtheorem{rem}[thm]{Remark}
\newtheorem{lem}[thm]{Lemma}

\font\myfont=cmr12 at 20pt

\title{\vspace{-5ex} \myfont Moduli of Complex Elliptic Curves}
\author{Jemin You}
\date{\vspace{-5ex}}
\maketitle

\setcounter{section}{0}

We follow Diamond and Shurman's book `A First Course on Modular Forms', Springer GTM 228.

\section{Complex Elliptic Curves}

\begin{defn}
A \textbf{complex elliptic curve} is a pointed compact Riemann surface of genus 1.
\end{defn}

It is well-known that an elliptic curve is isomorphic to a complex torus $\mathbb{C}/\Lambda$ where $\Lambda$ is a lattice of rank $2$.
The distingushed point corresponds to $0$.

\begin{thm}
Any  elliptic curve is isomorphic to a complex torus of the form
\[
\mathbb{C}/\Lambda_{\tau} \enskip (\Lambda_{\tau}:=\mathbb{Z}+\mathbb{Z}\tau, \enskip \textnormal{Im}\tau > 0).
\]
Moreover, this $\tau$ is unique up to the action of $\textnormal{SL}(2,\mathbb{Z})$ on the upper half-plane $\mathfrak{H}=\{\textnormal{Im}\tau>0\}$ given by
\[
\begin{bmatrix} a & b \\ c & d \end{bmatrix}.\tau = \frac{a\tau+b}{c\tau+d}.
\]
Hence the isomorphism classes of elliptic curves are in one-to-one correspondence with the points of the quotient space
\[
\mathfrak{H}/\textnormal{SL}(2,\mathbb{Z}) \stackrel{j}{\simeq} \mathbb{C},
\]
where $j$ is a weight $0$ modular function known as the $j$-invariant of the elliptic curve $\mathbb{C}/\Lambda_{\tau}$.
Here, if we write
\[
G_{2k}(\tau)=\sum_{\omega \in \Lambda_{\tau} \setminus \{0\}}\frac{1}{\omega^{2k}}, \enskip
g_2(\tau)=60G_4(\tau), \enskip g_3(\tau)=140G_6(\tau),
\]
then
\[
j=\frac{1728g_2^3}{g_2^3-27g_3^2}.
\]
\end{thm}

We call the space $\mathfrak{H}/\textnormal{SL}(2,\mathbb{Z})\simeq\mathbb{C}$ the $\textbf{moduli space}$ of elliptic curves for this reason.

\section{Compactifying the Moduli Space}

One of the issues of the moduli space $\mathbb{C}$ is that it is not $\textbf{compact}$.
We will compactify $\mathbb{C}$ by adding cusps.

\begin{defn}
The extended upper half-plane $\mathfrak{H}^*$ is a topological space defined as follows: as a set, it is $\mathfrak{H}\cup\mathbb{Q}\cup\{\infty\}$.
We then have an obvious extension of the action of $\textnormal{SL}(2,\mathbb{Z})$ on $\mathfrak{H}^*$.
The basic open sets for the topology are the open sets of $\mathfrak{H}$, and the sets
\[
\gamma.\{\textnormal{Im}\tau > \delta \}, \enskip \gamma \in \textnormal{SL}(2,\mathbb{Z}), \enskip \delta > 0.
\]
\end{defn}

The topology defined reflects some topological properties of actions near cusps.
One motivation of adding the rational and the infinity points to the upper half-plane is to think them of the limit points of the action.
We would like to explain the significance of the following theorem.

\begin{thm}
The quotient space $\mathfrak{H}^*/\textnormal{SL}(2,\mathbb{Z})$ has a structure of a compact Riemann surface.
It is isomorphic to $\mathbb{CP}^1$ and contain $\mathbb{C}=\mathfrak{H}/\textnormal{SL}(2,\mathbb{Z})$ as an open submanifold, of which the complement is the point $\infty$.
\end{thm}

The stabilizer of points of $\mathfrak{H}$ are generally $\{\pm$$I\}$, but orbits of $i$ and the third root of unity $\rho$ have stabilizers of order $4$ and $6$ respectively, and $\infty$ has a stabilizer group which is an extension of $\mathbb{Z}$ by $\pm$$I$.
The finite stabilizers mentioned above is not an issue by following well-known lemma from Riemann surface theory(take $G=\textnormal{PSL}(2,\mathbb{Z})$):
\begin{lem}
Let $X$ be a Riemann surface and $G$ be an abstract group.
If $G$ acts faithfully and holomorphically and properly discontinuously on $X$, the orbit space $X/G$ has a (unique) Riemann surface structure so the projection map $X$ $\to$ $X/G$ is holomorphic.
\end{lem}

What is interesting is that despite the infinite cyclic stabilizer of $\infty$ in $\textnormal{PSL}(2,\mathbb{Z})$ we can give chart near $\infty$ so that $\mathfrak{H}^*/\textnormal{SL}(2,\mathbb{Z})$ is a compactification of $\mathfrak{H}/\textnormal{SL}(2,\mathbb{Z})$.
This happens because some open sets of $\mathfrak{H}/\textnormal{SL}(2,\mathbb{Z})$ pulled back to $\mathfrak{H}$ `straightens' near $\infty$.

\begin{defn}
Let $X$ be a Riemann surface.
A $\textbf{hole chart}$ is a chart $\phi:U$ $\to$ $V$ from an open subset of $X$'s to $\mathbb{C}$'s such that there exists a closed subset $C$ of $X$ contained in $U$ that is mapped to a punctured disk of $\mathbb{C}$ contained in $V$.
\end{defn}

\begin{lem}
Maintain the notation of the above definition.
Then there exists a Riemann surface $X'$ which is set-theoretically $X$ with a point added.
The charts of $X'$ are the charts of $X$ and the chart obtained by the inverting extended $\phi^{-1}$, where the extended $\phi^{-1}$'s domain is the union of $V$ with the punctured point of the disk.
\end{lem}

So the picture is clear by now: the topology on $\mathfrak{H}^*$ described above is defined so the quotient space $\mathfrak{H}/\textnormal{SL}(2,\mathbb{Z})$ will have a hole chart `near $\infty$'.\\

The straightening can be observed by considering the behaviour of the $j$-invariant, which established the isomorphism $\mathfrak{H}/\textnormal{SL}(2,\mathbb{Z})\simeq\mathbb{C}$.
In fact, the $j$-invariant's Fourier expansion near $\infty$ is:
\[
j=\frac{1}{q}+744+196884q+\cdots \enskip (q=e^{2\pi i\tau}).
\]

Hence the open set $\{\textnormal{Im}\tau>\delta\}$ is mapped to $j(\{|q|<e^{-2\pi\delta}\})$, giving the hole chart punctured at $\infty$ for sufficient large $\delta$
(`disk punctured at $\infty$' is just $\{|z|>R\}$ for some $R>0$), because $j$ is injective modulo $1$ near $\infty$ by the fact

\[
\left(\frac{d}{dq}\right)_{q=0}\frac{1}{j}=\left(\frac{d}{dq}\right)_{q=0}(q-744q^2+\cdots)=1\neq0.
\]



\section{Enhanced Structures on Elliptic Curves}

One of the other issues about the moduli space $\mathfrak{H}/\textnormal{SL}(2,\mathbb{Z})$ is that it is not a $\textbf{fine moduli space}$.
Existence of nontrivial automorphisms of $\mathbb{C}/(\Lambda_{\tau})$ when $\tau=i$ or $\rho$ causes this failure.
Such can be overcome when we enhance the elliptic curves.

\begin{defn}
Let $E$ be an elliptic curve, and $N$ a natural number.
We denote by $E[N]$ the group of $N$-torsion points of $E$.
Fixing as isomorphism $E\simeq\mathbb{C}/\Lambda_{\tau}$, if $P$, $Q$ $\in$ $E[N]$, then we associate an $N^{th}$ root of unity $e(P,Q)$ defined by
\[
e(P,Q):=e^{\frac{2\pi i(ad-bc)}{N}} \textnormal{ where } P=\frac{a+b\tau}{N}, Q=\frac{c+d\tau}{N}.
\]
This is a well-defined bilinear pairing on $E[N]$ called the $\textbf{Weil pairing}$.
\end{defn}

The Weil pairing is intrinsic to $E$, which is not a priori clear.

\begin{defn}
Let $N$ be a natural number.
We define following level $N$ structures on elliptic curves.
\[
\begin{cases}
\Gamma_0(N)\textnormal{-structure on }E: & \textnormal{A cyclic subgroup of order }N.\\
\Gamma_1(N)\textnormal{-structure on }E: & \textnormal{A point of order }N.\\
\Gamma(N)\textnormal{-structure on }E: & \textnormal{A pair of generators of }E[N] \textnormal{ with Weil pairing } e^{\frac{2\pi i}{N}}.
\end{cases}
\]
hence a $\Gamma$-enhanced elliptic curve is a tuple of an elliptic curve with additional data, where $\Gamma$ is one of the above groups.
\end{defn}

The gamma groups above are exactly those from the class.
The results of the previous sections hold almost word in word.

\begin{thm}
Let $N$ be a natural number and $\Gamma$ be one of the groups $\Gamma_0(N)$, $\Gamma_1(N)$ or $\Gamma(N)$.
Then any $\Gamma$-enhanced elliptic curve is isomorphic to 
\[
\begin{cases}
(\mathbb{C}/\Lambda_{\tau},<\frac{1}{N}>), &\textit{if  }\Gamma=\Gamma_0(N).\\
(\mathbb{C}/\Lambda_{\tau},\frac{1}{N}), &\textit{if  }\Gamma=\Gamma_1(N).\\
(\mathbb{C}/\Lambda_{\tau},\frac{1}{N},\frac{\tau}{N}),  &\textit{if  }\Gamma=\Gamma(N).
\end{cases}
\]
Furthermore, this $\tau$ is unique up to  $\Gamma$-action.
Therefore, the moduli space of  $\Gamma$-enhanced elliptic curves is the quotient
\[
\mathfrak{H}/\Gamma
\]
which has a structure of a noncompact Riemann surface.
These are fine moduli spaces when $N>1$.
Such moduli spaces can be compactified by adding the cusps $\mathbb{Q}\cup\{\infty\}$, giving the compact Riemann surface
\[
\mathfrak{H}^*/\Gamma
\]
where the added points are $\Gamma$-orbits of $\mathbb{Q}\cup\{\infty\}$, which is a finite set since $[\textnormal{SL}(2,\mathbb{Z}):\Gamma]<\infty$.
\end{thm}




\end{document}