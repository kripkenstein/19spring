\documentclass[11pt,a4paper,reqno]{amsart} 
\usepackage{amsmath,amscd,amssymb,latexsym} 
\usepackage{longtable} 

\numberwithin{equation}{section} 

\def\baselinestretch{1.2} 

\newtheorem*{main}{Main Theorem} 
\newtheorem{thm}{Theorem}[section] 
\newtheorem{lem}[thm]{Lemma} 
\newtheorem{prop}[thm]{Proposition} 
\newtheorem{defn}[thm]{Definition} 
\newtheorem{cor}[thm]{Corollary} 
\newtheorem{ex}[thm]{Example} 
\newtheorem{rmk}[thm]{Remark} 

\pagestyle{plain} 

\newenvironment{pf}{{\noindent \bf Proof:\ }}{\hfill $\Box$ \bigskip} 

\renewcommand{\baselinestretch}{1.10}

\textwidth=17cm \textheight=25cm

\addtolength{\topmargin}{-40pt} \addtolength{\oddsidemargin}{-2cm}
\addtolength{\evensidemargin}{-2cm}

\setlength{\unitlength}{1mm} 



\begin{document} 


\title{Homework 1 : The ring of integers} 
\author{Hyojin Kim} 
\date{march 22, 2019} 
\maketitle 

\section{Definitions} 

In this section, we will introduce some basic definitions to know the notions of the ring of integer and the discriminant. 
%I brought out some parts of \cite{M} and \cite{S} but they were omitted in this note. You can check them in texfile. All are labelled with the exact location in the book for easy following.

\begin{defn}
Let $A$ be an integral domain, and let $L$ be a field containing $A$. 
An element $\alpha$ of $L$ is said to be integral over $A$ 
if it is a root if a monic polynomial with coefficients in $A$, 
i.e., of ot satisfies an equation 
\[ 
\alpha^n + a_1 \alpha^{n-1} + \cdots + a_n = 0,~a_i \in A. 
\] 
\end{defn}

An algebraic number field is a finite field extensions of $\mathbb{Q}$. 
The elements of the algebraic number field are called algebraic numbers. 
An algebraic number is called integral, or an algebraic integer, if it is a zero of a monic polynimial over $\mathbb{Z}$.

\begin{defn} \label{defn 2.5} 
The ring of elements of $L$ integral over $A$ is called the integral closure of $A$ in $L$. 
The integral closure of $\mathbb{Z}$ in an algebraic number field $L$ 
is called the ring of integers $\mathcal{O}_L$ in $L$. 
\end{defn} 

Consider for the field $\mathbb{Q}[\sqrt{D}]$, where $D$ is a square-free integer. 
Then the minimum polynomial of $a+b\sqrt{D} ,b\neq 0,a,b\in\mathbb{Q}$, is $x^2 - 2ax + (a^2 -b^2 D)$,
so $a+b\sqrt{D}$ is an algebraic integer
if and only if $2a \in \mathbb{Z}, (a^2 -b^2 D) \in \mathbb{Z}$.\\
(a) If $D \equiv 2$ or $D \equiv 3 \pmod 4$, $\mathcal{O}_{\mathbb{Q}[\sqrt{D}]}=\mathbb{Z}[\sqrt{D}]$ consists of all elements of the form $a+b \sqrt{D}$ with $a$, $b \in \mathbb{Z}$.\\
(b) If $D \equiv 1 \pmod 4 $, $\mathcal{O}_{\mathbb{Q}[\sqrt{D}]}=\mathbb{Z}[\frac{a+b \sqrt{D}}{2}]$.

\begin{defn} \label{defn 2.8} 
A ring $A$ is integrally closed 
if it is its own integral closure in its field of fractions $K$, 
i.e., if 
\[ 
\alpha \in K,~ \alpha ~integral~over A \Rightarrow \alpha \in A. 
\] 
\end{defn} 

%\begin{ex} \label{ex 2.10}
The rings $\mathbb{Z}$ and $\mathbb{Z}[i]$ are integrally closed, 
%because both are principal ideal domains. 
but unique factorization fails in $\mathbb{Z}[\sqrt{-3}]$.
%(c) For every field $k$, the integral closure of $k[S_1, \dots,S_m]$ in $k(X_1,\dots,X_m)$ is $k[X_1,\dots,X_m]$ 
%(here the $S_i$ are the elementary symmetric polynomials).\\ 
%\end{ex} 

\begin{defn} 
If $L$ is a finite extension of $K$ ($L$ and $K$ fields), then 
\[ 
(\alpha ,~ \beta ) \mapsto Tr_{L/K}(\alpha \beta ):L \times L \rightarrow K. 
\] 
is a symmetric bilinear form on $L$ regarded as a vector space over $K$, 
and the discriminant of this form is called the discriminant of $L/K$.\\ 
More generally, let $B \supset A$ be rings, and assume $B$ is free of rank $m$ as an $A$-module. 
Let $ \beta_1 ,\dots, \beta_m$ be elements of $B$. 
We define their discriminant to be 
\[ 
D(\beta_1 ,\dots, \beta_m )=det(Tr_{B/A}(\beta_i \beta_j )). 
\] 
In particular, the ideal in A that it generates is independent of the choice of the basis. 
This ideal, or $D( \beta_1,\dots,\beta_m )$ itself regarded as an elemet of $A/{A^{\times2}}$, 
is called the discriminant $disc(B/A)$ of $B$ over $A$. 
\end{defn} 

\begin{defn} \label{defn 2.32} 
When $K$ is a number field, a basis $\alpha_1, \dots, \alpha_m$ for $\mathcal{O}_K$ as a $\mathbb{Z}$-module is 
called an integral basis for $K$. 
\end{defn} 


\section{Properties}

\begin{thm}\label{m2.1} 
The elements of $L$ integral over $A$ form a ring. 
\end{thm} 

We will sketch of one of them which is Dedekind's. 
First, the following propsition is necessary. 

\begin{prop} \label{m2.4} 
Let $L$ be a field containg $A$. 
An element $\alpha$ of $L$ is integral over $A$ 
if and only if 
there exists a nonzero finitely generated $A$-submodule of $L$ 
such that 
$\alpha M \subset M$ 
(in fact, we can take $M=A[\alpha]$, 
the $A$-subalgebra generated by $\alpha$). 
\end{prop} 

One direction of the proof is trivial, and the opposite is using Cramer's rule. 
By using \ref{prop2.4}, \ref{thm2.1} is proved naturally. 

\begin{prop} \label{prop 2.9} 
A unique factorization domain, for example, a principal ideal domain, is integrally closed. 
\end{prop}
 

\begin{prop} \label{prop 2.11} 
Let $K$ be the field of fractions of $A$, 
and let $L$ be an extension of $K$ of finite degree. 
Assume $A$ is integrally closed. 
An element $\alpha$ of $L$ is integral over $A$ 
if and only if its minimum polynomial over $K$ has coefficients in $A$. 
\end{prop} 

%\begin{rmk} \label{rmk 2.12} 
An element $\alpha \in \mathbb{Q} [\sqrt d]$ is integral over $\mathbb{Z}$ 
if and only if its trace and norm both lie in $\mathbb{Z}$. 
%\end{rmk} 

\begin{prop} \label{prop 2.13} 
If $B$ is integral over $A$ and finitely generated as an $A$-algebra, 
then it is finitely generated as an $A$-module. 
\end{prop} 

%\begin{lem} \label{lem 2.14} 
%Let $A \subset B \subset C$ be rings. 
%If $B$ is finitely generated as an $A$-module, and $C$ is finitely generated as a $B$-module, 
%then $C$ is finitely generated as an $A$-module. 
%\end{lem} 

\begin{prop} \label{prop 2.15} 
Consider integral domains $A \subset B \subset C$ ; 
if $B$ is integral over $A$, and $C$ is integral over $B$, then $C$ in integral over $A$. 
\end{prop} 

\begin{cor} \label{cor 2.16} 
The integral closure of $A$ in an algebraic extension $L$ of its field of fractions is integrally closed. 
\end{cor} 

%\begin{rmk} \label{rmk 2.17} 
In particular, the ring of integers in a number field is integrally closed. 
%\end{rmk} 

%\begin{ex} \label{ex 2.18} 
Let $k$ be a finite field, and let $K$ be a finite extension of $k(X)$. 
Let $\mathcal{O}_K$ be the integral closure of $k[X]$ in $K$. 
The arithmetic of $\mathcal{O}_K$ is very similar to that of the ring of integers in a number field. 
%\end{ex} 


%\begin{lem} \label{lem 2.23} 
%If $\gamma_j=\sum{\alpha_{ji}}{\beta_i} ,~\alpha_{ij} \in A$, then 
%\[ 
%D(\gamma_1,\dots,\gamma_m )=det(a_{ij} )^2\cdot D(\beta _1,\dots, \beta_m ). 
%\] 
%\end{lem} 

\begin{prop} \label{prop 2.24} 
Let $A \subset B$ be integral domains and assume that $B$ is a free $A$-module of rank $m$ and that $disc(B/A)\neq 0$. 
Elements $\gamma_1,\dots ,\gamma_m$ form a basis for $B$ as an $A$-module if and only if 
\[ 
(D(\gamma_1,\dots,\gamma_m ))=(disc(B/A))~(as~ideals~in~A). 
\] 
\end{prop} 

%\begin{rmk} \label{rmk 2.25} 
Take $A=\mathbb{Z}$ in \ref{prop 2.24}. Elements $\gamma_1 , \dots, \gamma_m$ generate a submodule $N$ of finite index in $B$ 
if and only if $D(\gamma_1,\dots,\gamma_m )\neq 0$, in which case 
\[ 
D(\gamma_1,\dots,\gamma_m)=(B:N)^2 \cdot disc(B/\mathbb{Z}). 
\] 
%\end{rmk} 

%\begin{prop} \label{prop 2.26} 
%Let $L$ be a finite separable extension of the field $K$ of degree $m$, 
%and let $\sigma_1,\dots,\sigma_m$ be the distince $K$-homomorphisms of $L$ into some large Galois extension $\Omega$ of $L$. 
%Then, for any basis $\beta_1,\dots,\beta_m$ of $L$ over $K$, 
%\[ 
%D(\beta_1,\dots,\beta_m) = det(\sigma_i \beta_j )^2 \neq 0. 
%\] 
%\end{prop} 

%\begin{cor} \label{cor 2.27} 
%Let $K$ be the field of fractions of $A$, and let $L$ be a finite separable extension of $K$ of degree $m$. 
%If the integral closure $B$ of $A$ in $L$ is free of rank $m$ over $A$, then $disc(B/A) \neq 0$. 
%\end{cor} 

\begin{prop} \label{prop 2.29} 
Let $A$ be an integrally closed integral domain with field of fractions $K$, 
and let $B$ the integral closure of $A$ in a separable extension $L$ of $K$ of degree $m$. 
There exists free $A$-submodules $M$ and $M^\prime$ of $L$ 
such that 
\begin{equation} \label{6} 
M \subset B \subset M^\prime 
\end{equation} 
Therefore $B$ is a finitely generated $A$-module if $A$ is Noetherian, 
and it is free of rank $m$ if $A$ is a principal ideal domain. 
\end{prop} 

\begin{cor} \label{cor 2.30} 
The ring of integers in a number field $L$ is the largest subring that is finitely generated as a $\mathbb{Z}$-module. 
\end{cor} 

 $\mathcal{O}_K$ is finitely generated as a $\mathbb{Z}$-module. Furthermore, it will be a free module over $\mathbb{Z}$.

%We now assume $K$ to be a field of characteristic zero. 
%\begin{prop} \label{prop 2.34} 
%Let $L=K]\beta]$ some $\beta$, and let $f(X)$ be the minimum polynomial of $\beta$ over $K$. 
%Suppose that $f(X)$ factors into $\prod(X-\beta_i )$ over the Galois closure of $L$. Then 
%\[ 
%D(1,\beta,\beta^2,\dots,\beta^{m-1})=\prod_{1\le i < j \le m}(\beta_i -\beta_j )^2 = (-1)^{m(m-1)/2} \dot Nm_{L/K} (f^\prime (\beta) ). 
%\] 
%\end{prop} 

%The number in \ref{prop 2.34} is called the discriminant of $f(X)$. 



%\begin{prop} \label{prop 2.40} 
%Let $K$ be an algebraic number field.\\ 
%(a) The sign of $disc(K/\mathbb{Q})$ is $(-1)^s$, where $2s$ is the number of homomorphisms $K\hookrightarrow \mathbb{C}$ whose image is not contained in $\mathbb{R}$.\\ 
%(b) (Stickelberger's theorem) $disc(\mathcal{O}_K /\mathbb{Z} )\equiv 0$ or $1\pmod4$. 
%\end{prop} 


%\begin{prop} \label{prop 2.43} 
%Let $\beta_1,\dots,\beta_m$ be a basis for $L$ over $K$ consistion of elements of $B$, and let $d=disc(\beta_1,\dots,\beta_m)$. Then 
%\[ 
%A\cdot\beta_1 + \dots +A\cdot\beta_m \subset B \subset A\cdot(\beta_1 /d)+\dots +A\cdot (\beta_m /d). 
%\] 
%\end{prop} 

%\begin{lem} \label{lem 2.44} 
%Let $(A,\delta)$ be Euclidean domain, and let $M$ be an $m\times m$ matrix with coefficients in $A$. 
%Then it is possible to put $M$ into upper triangular form by elementary row operations of the following type:\\ 
%(r1) add a multiple of one row to a second;\\ 
%(r2) swap two rows. 
%\end{lem} 



\begin{thebibliography}{alpha} 
\bibitem{M} James. S. Milne, \emph{Algebraic Number Theory (v3.07)}, 2017. Available at www.jmilne.org/math/. 
\bibitem{S} P. Samuel, \emph{Algebraic Theory of Numbers}, traslated from the French by Allan J.Silberger, HERMANN, Paris, 1970.
\end{thebibliography} 

\end{document} 