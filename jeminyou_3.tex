\documentclass[a4paper,11pt]{article}


\usepackage{amsmath}
\usepackage{amssymb}
\usepackage{tikz-cd}
\usepackage{amsthm}
\usepackage{mathtools}
\usepackage{titling}
\usepackage[left=2cm,right=2cm,top=1cm,bottom=2cm,a4paper]{geometry}
\usepackage{mathrsfs}
\usepackage{calrsfs}



\begin{document}

\theoremstyle{plain}
\newtheorem{thm}{Theorem}[section]
\newtheorem{prp}[thm]{Proposition}
\newtheorem{lem}[thm]{Lemma}

\theoremstyle{definition}
\newtheorem{defn}[thm]{Definition}
\newtheorem{exm}[thm]{Example}
\newtheorem{nota}[thm]{Notation}

\theoremstyle{remark}
\newtheorem{rem}[thm]{Remark}



\font\myfont=cmr12 at 20pt

\title{\vspace{-5ex} \myfont Coarse Moduli Schemes \& Cusps}
\author{Jemin You}
\date{\vspace{-5ex}}
\maketitle

\setcounter{section}{-1}

We continue to follow Katz and Mazur, `Arithmetic Moduli of Elliptic Curves'.
We discuss coarse moduli schemes, the $j$-line and define the scheme of cusps.

\section{Brief Recall \& Geometric Properties of Moduli Problems}

Let $A$ be a ring.
Recall:

\begin{defn}
A moduli problem $\mathcal{P}$ on $Ell/A$ is $\textit{relatively representable}$ if for every elliptic curve $E/S$ over $A$ the moduli problem
\[
(Sch/S)^{\textnormal{op}} \to Sets, T \mapsto \mathcal{P}(E_T/T)
\]
is representable by a $S$-scheme, of which we denote by $\mathcal{P}_{E/S}$.

\end{defn}

All moduli problems on $Ell/A$ throughout this notes is assumed to be relatively representable.
Moduli problems on $Ell/A$ is considered a geometric object in more general context, and we can talk about geometric properties of them.

\begin{defn}
Let $\mathcal{P}$ be a relatively representable moduli problem on $Ell/A$.

(1) Let $Q$ be a property of morphism of schemes.
We say that $\mathcal{P}$ has property $Q$ if for every elliptic curve $E/S$ over $A$ ther morphism $\mathcal{P}_{E/S}\to$$S$ has property $Q$.

(2) Let $\textbf{E}/\mathfrak{M}$ be an elliptic curve over $A$.
We say that it is a $\textit{modular elliptic curve}$ over $A$ if the moduli problem it represents(=$\textnormal{Hom}_{Ell/A}(-,\textbf{E}/\mathfrak{M}$) is etale as a relatively representable moduli problem in the sense of (1).

(3) Let $R$ be a property of schemes local to the etale topology.
We say that $\mathcal{P}$ has property $R$ if for every modular elliptic curve $\textbf{E}/\mathfrak{M}$ over $A$ the scheme $\mathcal{P}_{\textbf{E}/\mathfrak{M}}$ has the property $R$.
\end{defn}

Here, $\textit{etale morphisms}$ of schemes are `analytically isomorphic finite coverings over an open set', and etale topology takes $\textit{etale morphisms as open neighbourhoods}$ instead of genuine Zariski open sets.
Modular elliptic curves thus can be thought of families of elliptic curves that has minimal autoomorphisms.

\begin{exm}\label{a}
(0) Possible candidates for $Q$ include $\textit{flat}$, $\textit{affine}$, $\textit{finite}$, etc.
Possible candidates for $R$ include $\textit{normal}$, $\textit{smooth over }\mathbb{Z}$, $\textit{regular of dimension }n$, etc.

(1) The moduli problems $[\Gamma]$ where $\Gamma$ is one of the groups $\Gamma(N)$, $\Gamma_1(N)$ or $\Gamma_0(N)$ from before are finite, flat((1) of the above) and regular of dimension $2$((3) of the above), as stated in the previous notes.

(2) The Legendre moduli problem(cf.previous notes) is etale.
\end{exm}

Hence we can talk geometrically of moduli problems on $Ell/A$, which is dealing systematically with structured elliptic curves varying in families.

\section{Compactification via Normalization at $j=\infty$}

\subsection{Coarse Moduli Schemes \& the $j$-line}

To any affine relatively representable moduli problem $\mathcal{P}$ on $Ell/A$, we always get a $\textit{coarse moduli scheme}$ $M(\mathcal{P})$ over $A$.
We will not define what it is here, but it should be thought of an $A$-scheme that is the `best replacement' of a non-representable moduli problems.
When the moduli problem is actually representable, it coincides with the base scheme of the representing elliptic curve.

One advantage of talking geometrically of moduli problems as done in the last section is that such language `descends down' to coarse moduli schemes.
For exmaple,
\[
\textnormal{If } \mathcal{P} \textnormal{ is normal, then so is } M(\mathcal{P}).
\]

\begin{rem}\label{b}
Since regular local rings are integrally closed, $[\Gamma]$ all are normal(cf. Example \ref{a}(1)).
\end{rem}

The phrasing `best replacement' can be slightly unambiguized by this result.

\begin{prp}
Let $\mathcal{P}$ be an affine relatively representable moduli problem on $Ell/A$, and let $k$ be an algebraically closed field.
Then there exists bijective identification
\[
M(\mathcal{P})(k) = k\textnormal{-isomorphism classes of } "elliptic \enskip curves \enskip E/k \enskip with \enskip \mathcal{P}\textnormal{-}structure".
\]
\end{prp}

Which implies that the coarse moduli schemes indeed bijectively encode the structured elliptic curves.
The failure of being representability comes from existence of nontrivial automorphism, as noted many times by now, therefore the structure being `not rich enough`.

\begin{exm}\label{c}
(1) Let $A=\mathbb{C}$ and consider the moduli problems $[\Gamma]$s over $\mathbb{C}$.
Then the coarse moduli schemes are the nonproper algebraic curves over $\mathbb{C}$,
\[
M([\Gamma])=Y(\Gamma)=\mathfrak{H}/\Gamma,
\]
i.e. the quotient of the complex upper-half plane by the congruence subgroup $\Gamma$ of $\textnormal{SL}(2,\mathbb{Z})$.

(2)Let $A$ be a $\mathbb{C}$-algebra.
Then the coarse moduli schemes for $[\Gamma]$ are the nonproper curves over $A$,
\[
Y(\Gamma) \otimes_\mathbb{C} A.
\]

\end{exm}

The simplest moduli problem we can consider is the problem $[\Gamma(1)]$.
It is the moduli problem sending each elliptic curve to the singleton set, which is the final object in the category of sets.
Therefore, if this was to be representable, it should be represented by a final object of $Ell/A$, which doesn't exist.
However, the situation is similar to the situation where we considered complex elliptic curves, where there was no fine moduli space but had the $j$-line $\mathbb{C}=\textnormal{Spec }\mathbb{C}[j]$ as coarse moduli space.
This translates analogously into general situations, giving
\[
\textnormal{the "}j \textnormal{"-line } \mathbb{A}^1_A=\textnormal{Spec }A[j]
\]
as the coarse moduli scheme for $[\Gamma(1)]$.

Being a final object of moduli problems, $[\Gamma(1)]$ admits a unique morphism(=a natural transformation) into itself from any moduli problem $\mathcal{P}$.
If $\mathcal{P}$ was affine, this gives a morphism
\[
M(\mathcal{P}) \to \textnormal{Spec }A[j]
\]
which should be thought of "sending the elliptic curves that a $\mathcal{P}$-structure was endowed to its $j$-value".

\begin{exm}[Recap of Example \ref{c}]
(1)The $\mathbb{C}$-morphism of coarse moduli schemes
\[
Y(\Gamma) \to \textnormal{Spec}\mathbb{C}[j]
\]
on closed points(ordinary points of the Riemann surfaces) is the forgetful map
\[
(E,\alpha \in \Gamma(E/\mathbb{C})) \mapsto j(E) \in \mathbb{C}
\]
($\alpha=\Gamma-$structure on E).

(2)If we take $A=\mathbb{C}[x,y]$, and consider the same map, we get
\[
Y(\Gamma) \otimes_{\mathbb{C}} \mathbb{C}[x,y] \to \textnormal{Spec }\mathbb{C}[x,y][j].
\]
Taking an $A$-section from the left gives us a family of two-parameter family of $\Gamma$-structured complex elliptic curves parametrized by $x$ and $y$.
Then the corresponding $A$-section obtained for the left side gives $j\in\mathbb{C}[x,y]$, a polynomial in $x$ and $y$, which outputs the $j$-value of the complex elliptic curve corresponding to chosen values $x_0$ and $y_0$ for each $(x_0,y_0)\in\mathbb{C}^2$.

\end{exm}

\subsection{Compactification of Coarse Moduli Schemes}

Let $A$ be an $\textit{excellent}$(=for e.g., rings of finite type over $\mathbb{Z}$ or a field) regular noetherian ring, and let a moduli problem $\mathcal{P}$ on $Ell/A$ satisfy the conditions

\begin{flushleft}
C1. $\mathcal{P}$ is relatively representable and finite over $Ell/A$.\\
C2. $M(\mathcal{P})$ is normal near $\infty$ in the sense that there exists a monic polynomial $f\in$$A[j]$ such that\\ $M(\mathcal{P})$ is normal over the locus where $f$ is invertible.
\end{flushleft}

\begin{rem}
Our moduli problems $[\Gamma]$ from above all satisfy these assumptions(cf.$\textit{Remark}$ \ref{b}).
\end{rem}

Then there exists a unique scheme $\overline{M}(\mathcal{P})$ that is $\textit{finite}$ over $\mathbb{P}^1_A$ extending $M(\mathcal{P})$ that is normal over the locus $f\neq0$.
It is the unique extension of $M(\mathcal{P})$ that is normal near $\infty$.
We consider that we've `added' the cusps over $\infty$.

\begin{defn}
The $\textit{scheme of cusps}$ of the moduli problem $\mathcal{P}$ satisfying the conditions C1. and C2. is defined to be the reduced scheme
\[
Cusps(\mathcal{P}) := (j^{-1}(\infty)))^{\textnormal{red}}.
\]
This is the reduced induced closed subscheme supported over $\infty$.
We denote the $\textit{formal completion}$ of $\overline{M}(\mathcal{P})$ along $Cusps(\mathcal{P})$ by
\[
\widehat{Cusps}(\mathcal{P}).
\]
\end{defn}

These are the analogues of the cusps that appeared in class in the general scheme-theoritic settings.
We haven't defined what $\textit{formal completion}$ is.
Intuitively, this is thought of `infinitesimal neighbourhood' of the closed subscheme; locally, the formal completion is the completion of a ring $R$ with respect to the $I$-adic topology generated by an ideal $I$:
\[
\widehat{R}:=\underleftarrow{ \vspace{0.4ex} \textnormal{lim}}R/I^n.
\]
An easy example can be given: the formal completion of the polynomial ring $\mathbb{C}[x_1,x_2,\cdots,x_n]$ in $n$ variables along the ideal $(x_1,x_2,\cdots,x_n)$ is $\mathbb{C}[[x_1,x_2,\cdots,x_n]]$, the ring of formal power series in $n$ variables.
This contains the ring of germ of holomorphic functions at $0$, which corresponds to the ideal above.
Therefore, what being done here can be roughly thought of considering `holomorphic neighbourhoods' of cusps.

\begin{exm}[Taken from `97 paper of Hibino and Murayabashi]
Let $A=\mathbb{C}$.
Then the compactified coarse moduli scheme is $X(\Gamma)$.
For explicit calculations, let $\Gamma=\Gamma_0(37)$.
Then there are $2$ cusps:
\[
\frac{a}{c}=\frac{0}{1},\frac{1}{0}.
\]
and one can calculate:
\[
X_0(37)=\overline{\{ y^2=1+14x+35x^2+48x^3+35x^4+14x^5+x^6\}} \subseteq \mathbb{CP}^2.
\]
Moreover, the $j$-invariant is expressed in the form
\[
j=(A(x)-B(x)y)/2x^{37}
\]
where $A$, $B$ are monic polynomial with constant term $1$ of degree $38$, $35$ respectively, i.e.,
\[
X_0(37) \ni (x,y) \mapsto [2x^37:A(x)-B(x)y] \in \mathbb{CP}^1
\]
is the map into the $j$-line, and the cusps are exactly
\[
(0,\pm1)
\]
two of them, as expected(not sure which is $0$ and which is $\infty$).
Hence the scheme $Cusps([\Gamma_0(37)])$ is a reduced two-point scheme over $\mathbb{C}$.
In this case taking formal completion seems inefficient, since we already understand the cusp itself.
\end{exm}


Next week, we will discuss the $\textit{Tate curve}$, which is an elliptic curve over the ring of finite-tailed Laurent series with integer coefficients.
For relatively representable moduli problems satisfying certain conditions($[\Gamma]$s do satisfy these), it will be stated that the computations of $\widehat{Cusps}(\mathcal{P})$ is essentially computation of the relative representative over the Tate curve(=$\mathcal{P}_{(\textnormal{Tate Curve over }A)}$).

On the last weeks, we will then actually go through an explicit computation for a moduli problem.
However, the computations that will be done will be radically general, in contrast to previous examples all done over $\mathbb{C}$.







\end{document}























































