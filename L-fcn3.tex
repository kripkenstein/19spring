\documentclass[11pt]{article}
\usepackage[left=3cm,right=3cm,top=3cm,bottom=3cm,a4paper]{geometry}
\usepackage{amssymb, amsmath}

\title{L-function and the theorem on arithmetic progressions}
\author{Kangsig Kim}

\begin{document}

\maketitle
\textbf{Abstract.} We define the Dirichlet L-function and use its properties to prove that there exist infinitely many prime numbers $p$ such that $p\equiv a$ (mod $m$) where $a$ and $m$ are relatively prime integers $\geq1$.
\vspace{10mm}

\section{Dirichlet series}
Let $(\lambda_n)$ be an increasing sequence of real numbers tending to infinity. A \textit{Dirichlet series} with exponents $(\lambda_n)$ is a series of the form
\begin{equation*}
    f(z)=\displaystyle\sum_{n=1}^{\infty} a_n e^{-\lambda_n z} \quad (a_n\in \mathbb{C}, z\in \mathbb{C})
\end{equation*}
\\
These are the properties of Dirichlet series which can be proved by using the theories of complex analysis.
\\
\\
\textbf{Proposition 1.}
If $f$ converges for $z=z_0$, it converges for $\Re(z)>\Re(z_0)$ and it is holomorphic in that domain.
\vspace{7mm}
\\
\textbf{Proposition 2.}
Let $a_n$ are real $\geq0$. Suppose that $f$ converges for $\Re(z)>\rho$ and that $f$ can be extended analytically to a function holomorphic in a neighborhood of the point $z=\rho$. Then there exists $\epsilon>0$ such that $f$ converges for $\Re(z)>\rho-\epsilon$.
\vspace{8mm}

When $\lambda_n=\log n$, we get $f(s)=\displaystyle\sum_{n=1}^{\infty} {a_n\over n^s}$, which is a form of the zeta function and L-function. The notation $s$ being traditional for the variable.
\\
\\
\textbf{Proposition 3.}
If $a_n$ are bounded, then $f(s)=\displaystyle\sum_{n=1}^{\infty} {a_n\over n^s}$ converges absolutely for $\Re(s)>1$.
\par
This follows from the convergence of $\displaystyle\sum_{n=1}^{\infty} {1\over n^\alpha}$ for $\alpha>1$, $\alpha\in\mathbb{R}$
\vspace{3mm}
\\
\textbf{Proposition 4.}
If every partial sum $\displaystyle\sum_{n=m}^{n=p} a_n$ is bounded, then $f(s)=\displaystyle\sum_{n=1}^{\infty} {a_n\over n^s}$ converges (not necessarily absolute) for $\Re(s)>0$.
\vspace{10mm}

\section{Zeta function}
In the following, $P$ denotes the set of prime numbers. Recall the properties of the zeta function $\zeta(s)=\displaystyle\sum_{n=1}^{\infty} {1\over n^s}=\displaystyle\prod_{p\in P} {1\over 1-p^{-s}}$, which equalities holds for $\Re(s)>1$.
\vspace{7mm}
\\
\textbf{Proposition 5.}
(a) $\zeta(s)$ is holomorphic and nonzero for $\Re(s)>1$. \\
(b) $\zeta(s)={1\over s-1}+\phi(s)$, where $\phi(s)$ is holomorphic for $\Re(s)>0$. Thus $\zeta(s)$ extends analytically for $\Re(s)>0$ and has a simple pole at $s=1$.
\\
\\
\textbf{Proposition 6.}
When $s\rightarrow 1$, one has $\displaystyle\sum_{p\in P}p^{-s}\thicksim \log\:1/(s-1)$, and $\displaystyle\sum_{p\in P, k\geq2}1/p^{ks}$ remains bounded.
\\
\\
Proof) Using that $\log(1-z)=-(z+{z^2\over2}+{z^3\over3}+...)$ for $|z|<1$, one has:
\begin{equation*}
\log\zeta(s)=\log\displaystyle\prod_{p\in P} {1\over 1-p^{-s}}=\displaystyle\sum_{p\in P}-\log(1-p^{-s})=\displaystyle\sum_{p\in P,\; k\geq1}{1\over k\cdot p^{ks}}=\displaystyle\sum_{p\in P}1/p^s+\psi(s)
\end{equation*}
where $\psi(s)=\displaystyle\sum_{p\in P,\; k\geq2}(1/k\cdot p^{ks})$. The series $\psi(s)$ is majorized by
\begin{equation*}
\displaystyle\sum_{p\in P,\; k\geq2}1/p^{ks}=\sum{1/p^s(p^s-1)}\geq\sum1/p(p-1)\geq\displaystyle\sum_{n=2}^{\infty}1/n(n-1)=1.
\end{equation*}
Thus $\psi(s)$ is bounded, and since proposition 5(b) shows that $\log\zeta(s)\thicksim \log\:1/(s-1)$ as $s\rightarrow 1$, we get $\displaystyle\sum_{p\in P}p^{-s}\thicksim \log\:1/(s-1)$.
\vspace{10mm}

\section{Characters of finite abelian groups and L-functions}
Let $G$ be a finite abelian group. A \textit{character} of $G$ is a homomorphism of $G$ into the multiplicative group $\mathbb{C}^*$ of complex numbers. The characters of $G$ form a group $Hom(G,\mathbb{C}^*)$ which we denote by $\hat{G}$ and call the \textit{dual} of $G$. Note that the group $\hat{G}$ is also a finite abelian group of the same order as $G$. For $\chi\in\hat{G}$ and $x\in G$, we have $|\chi(x)|=1$ because $\chi(x)^n=\chi(x^n)=\chi(1)=1$ where $n$ is the order of $x$.
\vspace{5mm}
\\
\textbf{Proposition 7.}
Let $n$ be the order of $G$ and let $\chi\in\hat{G}$. Then
\begin{equation*}
    \displaystyle\sum_{x\in G} \chi(x)=\left \{\begin{array}{l}
    n,\quad if \; \chi=1 \\
    0,\quad if \; \chi\neq 1
    \end{array}
    \right.
\end{equation*}
Proof) The first formula is obvious. To prove the second, choose $y\in G$ such that $\chi(y)\neq 1$. Then $\chi(y)\displaystyle\sum_{x\in G} \chi(x)=\displaystyle\sum_{x\in G} \chi(xy)=\displaystyle\sum_{x\in G} \chi(x)$, hence $(\chi(y)-1)\displaystyle\sum_{x\in G} \chi(x)=0$. Since $\chi(y)\neq 1$, this implies $\displaystyle\sum_{x\in G} \chi(x)=0$.
\vspace{10mm}
\\
\textbf{Proposition 8.}
Let $x \in G$. Then
\begin{equation*}
    \displaystyle\sum_{\chi\in \hat{G}} \chi(x)=\left \{\begin{array}{l}
    n,\quad if \; x=1 \\
    0,\quad if \; x\neq 1.
    \end{array}
    \right.
\end{equation*}
This follows from Proposition 7 applied to the dual group $\hat{G}$.
\vspace{10mm}

Let $m\geq 1$ be an integer. We let $(\mathbb{Z}/m\mathbb{Z})^*$ the multiplicative group of invertible elements of the ring $\mathbb{Z}/m\mathbb{Z}$ and let $\chi$ be a character of $(\mathbb{Z}/m\mathbb{Z})^*$. We can view $\chi$ as a multiplication function, defined on the set of integers prime to $m$, with values in $\mathbb{C}$. We extend the domain of the function to whole $\mathbb{Z}$ by putting $\chi(a)=0$ if $a$ is not prime to $m$.
\\
\\
The corresponding \textit{L-function} is defined by the Dirichlet series
\begin{center}
    $L(s,\chi)=\displaystyle\sum_{n=1}^{\infty} \chi(n)/n^s$
\end{center}
\vspace{2mm}

\noindent \textbf{Proposition 9.}
For $\chi=1$, we have $L(s,1)=F(s)\zeta(s)$ with $F(s)=\displaystyle\prod_{p|m} (1-p^{-s})$. \\ In particular $L(s,1)$ extends analytically for $\Re(s)>0$ and has a simple pole at $s=1$.
\vspace{10mm}
\\
\textbf{Proposition 10.}
For $\chi\neq 1$, the series $L(s,\chi)$ converges in $\Re(s)>0$ and converges absolutely in $\Re(s)>1$; one has
\begin{equation*}
    L(s,\chi)=\displaystyle\prod_{p\in P} {1\over 1-\chi(p)p^{-s}} \quad for \quad \Re(s)>1
\end{equation*}
Proof) The assertion for absolute convergence in $\Re(s)>1$ follows from proposition 3. Thus in $\Re(s)>1$, we get a series of equalities
\begin{equation*}
   L(s,\chi)=\displaystyle\sum_{n=1}^{\infty} \chi(n)/n^s=\displaystyle\prod_{p\in P}\bigg(\displaystyle\sum_{m=0}^{\infty} \chi(p^m)/p^{ms}\bigg)=\displaystyle\prod_{p\in P} {1\over 1-\chi(p)p^{-s}}
\end{equation*}.
Here, we used that $\chi(ab)=\chi(a)\chi(b)$ for every $a,b\in \mathbb{Z}$. It remains to show the convergence of $L(s,\chi)$ for $\Re(s)>0$. Using proposition 4, it suffices to show that $\displaystyle\sum_{n=u}^{n=v} \chi(n)$ are bounded. By proposition 7, we have $\displaystyle\sum_{n=u}^{n=u+m-1} \chi(n)=0$. Thus it suffices to majorize $\displaystyle\sum_{n=u}^{n=v} \chi(n)$ for $v-u<m-1$. But since $|\chi(x)|=1$, one has $|\displaystyle\sum_{n=u}^{n=v} \chi(n)|\leq \phi(m)$ where $\phi(m)$ is an order of the group $(\mathbb{Z}/m\mathbb{Z})^*$, given by the Euler $\phi$-function of $m$. This completes the proof.
\vspace{15mm}

\section{The theorem of non-vanishing L-functions}
The key point of Dirichlet's proof is to show that $L(1,\chi)\neq 0$ for all $\chi\neq 1$. To show it, we introduce the product of the L-functions relative to the same integer $m$.
\\
\par Let $m\geq 1$ be an integer. If prime number $p$ does not divide $m$, we let $f(p)$ the order of $p$ in $(\mathbb{Z}/m\mathbb{Z})^*$. That is, $f(p)$ is the smallest integer $f>1$ such that $p^f\equiv 1$ (mod $m$). We put $g(p)=\phi(m)/f(p)$.
\vspace{5mm}
\\
\textbf{Lemma 1.}
If $p\nmid m$, then $\displaystyle\prod_{\chi}(1-\chi(p)T)=(1-T^{f(p)})^{g(p)}$, where the product extends over all characters $\chi$ of $(\mathbb{Z}/m\mathbb{Z})^*$.
\\
\\
Proof) Let $W$ be the set of $f(p)$-th roots of unity. Then 
\begin{multline*}
T^{f(p)}-1=\displaystyle\prod_{w\in W}(T-w)=(-1)^{f(p)}\displaystyle\prod_{w\in W}(w-T)
\\=(-1)^{f(p)}\Big(\displaystyle\prod_{w\in W}w\Big)\Big(\displaystyle\prod_{w\in W}(1-w^{-1}T\Big)
=-\displaystyle\prod_{w\in W}(1-w^{-1}T)=-\displaystyle\prod_{w\in W}(1-wT),
\end{multline*}
so we get $\displaystyle\prod_{w\in W}(1-wT)=1-T^{f(p)}$. Now for each $w\in W$ there exists $g(p)$ characters $\chi$ of $(\mathbb{Z}/m\mathbb{Z})^*$ such that $\chi(p)=w$. Hence $\displaystyle\prod_{\chi}(1-\chi(p)T)=(1-T^{f(p)})^{g(p)}$ holds.
\vspace{6mm}

We define a function $\zeta_m(s)$ by
\begin{equation*}
    \zeta_m(s)=\displaystyle\prod_{\chi} L(s,\chi),
\end{equation*}
where the product extends over all characters $\chi$ of $(\mathbb{Z}/m\mathbb{Z})^*$.
\vspace{7mm}
\\
\textbf{Proposition 11.}
One has \begin{equation*}
    \zeta_m(s)=\displaystyle\prod_{p\:\nmid\: m} {1\over (1-p^{-f(p)s} )^{g(p)}}.
\end{equation*}
This is a Dirichlet series, with positive integral coefficients, converging in $\Re(s)>1$.
\vspace{6mm}
\\
Proof) Replacing each L-function by its product expansion, and applying lemma 1 (with $T=p^{-s}$) noticing that $\chi(p)=0$ if $p$ divides $m$, we obtain the product expansion of $\zeta_m(s)$. The assertion for convergence in $\Re(s)>1$ follows from proposition 9, 10 and the equality $\zeta_m(s)=\displaystyle\prod_{\chi} L(s,\chi)$. And since
\begin{equation*}\zeta_m(s)=\displaystyle\prod_{p\:\nmid\: m} {1\over (1-p^{-f(p)s})^{g(p)}}=\displaystyle\prod_{p\:\nmid\: m}(1+{1\over p^{f(p)s}}+{1\over p^{2f(p)s}}+{1\over p^{3f(p)s}}+...)^{g(p)}\quad for\quad \Re(s)>1,
\end{equation*}
we deduce that $\zeta_m(s)$ is a Dirichlet series with positive integral coefficients.
\\
\\
\textbf{Theorem 1.} \textit{$L(1, \chi)\neq0$ for all $\chi\neq1$.} \\
\\
Proof) Suppose that $L(1, \chi)=0$ for some $\chi\neq1$. Then $\zeta_m(s)$ would be holomorphic at $s=1$, thus also for all $s$ in $\Re(s)>0$ by proposition 9, 10. Since $\zeta_m(s)$ is a Dirichlet series with positive coefficients, it would converge for all $s$ in the same domain by proposition 2. (take $\rho$ be very small positive real number in proposition 2) \\
However, the $p$-th factor of $\zeta_m(s)$ is
\begin{equation*}
{1\over (1-p^{-f(p)s})^{g(p)}}=(1+p^{-f(p)s}+p^{-2f(p)s}+...)^{g(p)}
\end{equation*}
and it dominates $1+p^{-\phi(m)s}+p^{-2\phi(m)s}+...$.\\
It follows that $\zeta_m(s)$ dominates the series $\displaystyle\sum_{(n, m)=1}n^{-\phi(m)s}$ which diverges for $s={1\over \phi(m)}$.
\vspace{2mm}

\noindent It contradicts the convergence of $\zeta_m(s)$ in $\Re(s)>0$. This completes the proof.
\\
\\
\par Define the series:
\begin{equation*}
f_{\chi}(s)=\displaystyle\sum_{p\:\nmid\: m}{\chi(p)\over p^s}
\end{equation*}
This series being convergent for $\Re(s)>1$ by proposition 3. We get essential properties of $f_{\chi}(s)$ for $s\to 1$ by using theorem 1.
\vspace{7mm}
\\
\textbf{Proposition 12.}
If $\chi=1$, then $f_{\chi}(s) \thicksim \log{1\over s-1}$ for $s\to 1$
\\
\par
This follows from proposition 6 and the fact that $f_1$ differs from the series $\displaystyle\sum_{p\in P}{1\over p^s}$ by a finite number of terms only.
\vspace{7mm}
\\
\textbf{Proposition 13.}
If $\chi\neq 1$, then $f_{\chi}(s)$ remains bounded when $s\to 1$.
\\
\\
Proof) Use again the identity $\log(1-z)=-(z+{z^2\over2}+{z^3\over3}+...)$ for $|z|<1$.
\\
Then for $\Re(s)>1$, one has:
\begin{equation*}
\log L(s, \chi)=\displaystyle\sum_{p\:\nmid\: m}\log{1\over 1-\chi(p)p^{-s}}=\displaystyle\sum_{n\geq 1, p\:\nmid\: m}{\chi(p)^n \over np^{ns}}=f_{\chi}(s)+F_{\chi}(s) 
\end{equation*}
with $F_{\chi}(s)=\displaystyle\sum_{n\geq 2, p\:\nmid\: m}{\chi(p)^n \over np^{ns}}$.
\vspace{2mm}
\\
Theorem 1, proposition 6, 10 shows that $\log L(s, \chi)$ and $F_{\chi}(s)$ remain bounded when $s\to 1$. Hence the same holds for $f_{\chi}(s)$.
\vspace{5mm}
\\
\par
Now we ready to prove the theorem on arithmetic progressions.
\newpage
\section{Density and Dirichlet theorem}
Recall that when $s$ tends to 1 (let $s$ being real $>1$ to fix the ideas),
\\
one has $\displaystyle\sum_{p\in P}{1\over p^s}\thicksim \log{1\over s-1}$. Let $A$ be a subset of $P$. One says that $A$ has for \textit{density} a real number $k$ when the ratio
\begin{equation*}
\bigg(\displaystyle\sum_{p\in A}{1\over p^s}\bigg) \bigg/ \bigg(\log{1\over s-1}\bigg)
\end{equation*}
tends to $k$ when $s\to 1$. One then has $0\leq k \leq 1$. The theorem on arithmetic progressions can be refined in the following way:
\\
\\
\textbf{Theorem 2.}
Let $m\geq1$ and $a$ be an integer such that $(a, m)=1$. Let $P_a$ be the set of prime numbers such that $p\equiv a\; ($mod $m)$. The set $P_a$ has density $1/\phi(m)$.
\\
(In other words, the prime numbers are "equally distributed" between the different classes modulo $m$ which are prime to $m$.)
\\
\\
Proof) Define the series
\begin{equation*}
g_a (s)=\displaystyle\sum_{p\in P_a} 1/p^s
\end{equation*}
We claim that $g_a (s)={1\over \phi(m)}\displaystyle\sum_{\chi}\chi(a)^{-1}f_\chi(s)$, where the sum extends over all characters $\chi$ of $(\mathbb{Z}/m\mathbb{Z})^*$. Observe that $\displaystyle\sum_{\chi}\chi(a)^{-1}f_\chi(s)=\displaystyle\sum_{p\:\nmid\: m}\Big(\displaystyle\sum_{\chi}\chi(a^{-1})\chi(p)\Big)/p^s$
\\
and $\chi(a^{-1})\chi(p)=\chi(a^{-1}p)$. By proposition 8, we have:
\begin{equation*}
    \displaystyle\sum_{\chi} \chi(a^{-1}p)=\left \{\begin{array}{l}
    \phi(m),\quad if \;\: a^{-1}p\equiv 1\:\, (mod\:\, m) \\
    0,\quad otherwise. 
    \end{array}
    \right.
\end{equation*}
Thus the claim is proved. Now, as $s$ tends to 1, proposition 12 shows that $f_{\chi}(s) \thicksim \log{1\over s-1}$ for $\chi=1$, and proposition 13 shows that all other $f_\chi$ remain bounded. Put them in $g_a (s)={1\over \phi(m)}\displaystyle\sum_{\chi}\chi(a)^{-1}f_\chi(s)$, we get $g_a(s)\thicksim {1\over \phi(m)}\log{1 \over s-1}$ and this means that the density of $P_a$ is ${1\over \phi(m)}$.
\vspace{9mm}
\\
\textbf{Theorem 3 (Dirichlet).}
There exist infinitely many prime numbers $p$ such that $p\equiv a$ (mod $m$) where $a$ and $m$ are relatively prime integers $\geq1$.
\\
\\
Proof) This is a corollary of theorem 2. Since the density of $P_a$ is ${1\over \phi(m)}$ which is a positive real number, the set $P_a$ is infinite. Indeed a finite set has density zero.
\vspace{10mm}

\section*{References}
[1] J.-P.Serre, \textit{A Course in Arithmetic}, Springer, 1973

\end{document}